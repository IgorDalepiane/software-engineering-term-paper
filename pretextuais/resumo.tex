\begin{resumo}

\textbf{Contexto}: Devido às diretrizes impostas pela Resolução Nº 7 de 2018 do Conselho Nacional de Educação (CNE) \cite{Resolucao-MEC:2018}, a curricularização da extensão se tornará obrigatória no ano de 2023. Com isso, os cursos de graduação de todas as universidades federais deverão alocar uma carga horária equivalente a 10\% (dez por cento) da total do curso, incentivando os alunos a procurarem por mais atividades extensionistas e aos docentes a proporem mais atividades.
\textbf{Objetivo}: Tendo em vista que os processos que dizem a respeito da criação e manutençao de atividades de extensão são burocraticos, demorados e manuais, o objetivo deste trabalho é produzir uma ferramenta baseada na Web, capaz de dar suporte em todos ou na grande maioria destes processos. O seu desenvolvimento está sendo realizado por dois alunos de graduação, dividindo a carga de trabalho que evolvem todo o processo de Engenharia de Software em \textit{frontend} e \textit{backend}, este trabalho se concentra na área do \textit{backend}.
\textbf{Método}: Para isso ser possível, dois métodos científicos foram utilizados:
uma revisão sistematica na literatura cinza, com o objetivo de encontrar ferramentas semelhantes, para a coleta de requisitos e detalhes pertinentes. O segundo foi um levantamento (\textit{survey}) com os possíveis usuários do sistema que fazem parte da comunidade acadêmica da UNIPAMPA. O objetivo foi classificar por ordem de importância as funcionalidades e, além disso, permitindo com que os participantes fornecessem sugestões relacionadas as mesmas, ou até que sugerissem novas. 
\textbf{Resultados}: Com os resultados obtidos, foi possivel construir uma lista de requisitos classificados pela sua prioridade, junto com funcionalidades adicionais sugeridas pelos respondentes do questionário. Sendo assim, o desenvolvimento da ferramenta já está direcionado e pode ser iniciado.
\textbf{Conclusão}: Olhando para a hipótese levantada por este estudo, não é ainda possível confirma-la ou refuta-la, pois a ferramenta ainda não foi desenvolvida e testada com os usuários finais. Entretanto, com os resultados positivos obtidos através do levantamento (\textit{survey}) e da revisão na literatura cinza, é bem provável que ela seja confirmada, através do desenvolvimento correto e completo da aplicação final.

% Devido às diretrizes impostas pela Resolução Nº 7 de 2018 do Conselho Nacional de Educação (CNE) \cite{Resolucao-MEC:2018}, a curricularização da extensão se tornará obrigatória no ano de 2023. 
% Tendo em vista que o processo de criação e manutenção de um programa ou projeto de extensão é demasiado demoroso e adicionado com a obrigatoriedade eminente, o objetivo deste trabalho é implementar uma ferramenta que ofereça suporte ao processo como um todo, permitindo desde a criação até a emissão de certificados para os participantes. 
% Para isto ser possível, realizou-se primeiramente uma revisão sistemática na literatura cinza, em busca de ferramentas semelhantes, sendo extraído funcionalidades e aspectos mais pertinentes entre estas. 
% Posteriormente, utilizando a lista alcançada, foi executado um levantamento (\textit{survey}) com os possíveis usuários finais da comunidade acadêmica da UNIPAMPA, com objetivo de classificar por ordem de importância as funcionalidades, além disso, permitindo com que os participantes fornecessem sugestões relacionadas as mesmas, ou até mesmo sugerindo novas. 
% Os resultados foram analisados e iniciou-se a produção da solução proposta, uma ferramenta baseada na Web que auxiliará no esforço manual requerido nos processos relacionados a atividades de extensão. 
% O desenvolvimento desta proposta de solução está sendo realizado por dois alunos de graduação, dividindo a carga de trabalho que evolvem todo o processo de Engenharia de Software em \textit{frontend} e \textit{backend}, este trabalho se concentra na área do \textit{backend}.

\vspace{\onelineskip}
    
\noindent
\textbf{Palavras-chave}: Ferramenta. \textit{Survey}. Literatura Cinza. \textit{Backend}. NestJS. Extensão. Atividade Extensionista. Comunidade. Universidade.

\end{resumo}
