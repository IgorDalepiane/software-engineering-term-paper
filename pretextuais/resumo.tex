\begin{resumo}

Devido às diretrizes impostas pela Resolução Nº 7 de 2018 do Conselho Nacional de Educação (CNE) \cite{Resolucao-MEC:2018}, a curricularização da extensão se tornará obrigatória no ano de 2023. 
Tendo em vista que o processo de criação e manutenção de um programa ou projeto de extensão é demasiado demoroso e adicionado com a obrigatoriedade eminente, o objetivo deste trabalho é implementar uma ferramenta que ofereça suporte ao processo como um todo, permitindo desde a criação até a emissão de certificados para os participantes. 
Para isto ser possível, realizou-se primeiramente uma revisão sistemática na literatura cinza, em busca de ferramentas semelhantes, sendo extraído funcionalidades e aspectos mais pertinentes entre estas. 
Posteriormente, utilizando a lista alcançada, foi executado um levantamento (\textit{survey}) com os possíveis usuários finais da comunidade acadêmica da UNIPAMPA, com objetivo de classificar por ordem de importância as funcionalidades, além disso, permitindo com que os participantes fornecessem sugestões relacionadas as mesmas, ou até mesmo sugerindo novas. 
Os resultados foram analisados e iniciou-se a produção da solução proposta, uma ferramenta baseada na Web que auxiliará no esforço manual requerido nos processos relacionados a atividades de extensão. 
O desenvolvimento foi realizado por dois alunos de graduação, dividindo a carga em frontend e backend, este trabalho se concentra na área do backend.

\vspace{\onelineskip}
    
\noindent
\textbf{Palavras-chave}: Ferramenta. Survey. Literatura Cinza. Backend. Extensão. Atividade Extensionista. Comunidade. Universidade.

\end{resumo}
