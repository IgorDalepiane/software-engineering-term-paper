\begin{resumo}[Abstract]
 Due to the guidelines imposed by Resolution No. 7 of 2018 of the National Education Council (CNE) \cite{Resolucao-MEC:2018}, the curricularization of the extension will become mandatory in 2023. Given that the process of creation and maintenance of an extension program or project is too time-consuming and added to the imminent obligation, the objective of this work is to implement a tool that supports the process as a whole, allowing from the creation to the issuance of certificates for the participants. For this to be possible, a systematic review was first carried out in the gray literature, in search of similar tools, extracting the most relevant features and aspects among them. Subsequently, using the list reached, a survey was carried out with the possible end users of the academic community of UNIPAMPA, with the objective of classifying the functionalities in order of importance, in addition, allowing the participants to provide suggestions related to them, or even suggesting new ones. The results were analyzed and the production of the proposed solution began, a web tool that will assist in the manual effort required in the processes related to extension activities. The development was carried out by two undergraduate students, dividing the load into front-end and back-end, this work focuses on the back-end area.

 \vspace{\onelineskip}
 
 \noindent 
 \textbf{Key-words}: Tool. Survey. Grey Literature. Backend. Outreach. Community. University.
\end{resumo}
