\begin{resumo}[Abstract]

\textbf{Context}: Due to the guidelines imposed by Resolution No. 7 of 2018 of the National Council of Education (CNE) \cite{Resolucao-MEC:2018}, the curricularization of outreach will become mandatory in 2023. With this, the undergraduate courses of all federal universities must allocate a workload equivalent to 10\% (ten percent) of the total course, encouraging students to look for more outreach activities and professors to propose more.
\textbf{Objective}: Considering that the processes related to the creation and maintenance of outreach activities are bureaucratic, time-consuming and manual, the objective of this work is to produce a web-based tool, capable of supporting all or the vast majority of these processes. Its development is being carried out by two undergraduate students, dividing the workload involved in the entire Software Engineering process into frontend and backend, this work focuses on the backend area.
\textbf{Method}: To make this possible, two scientific methods were used:
the first is a systematic review in the gray literature, with the objective of finding similar tools, for the collection of requirements and pertinent details. The second was a survey with the possible users of the system that are part of the academic community of UNIPAMPA. The objective was to classify the requirements in order of importance and, in addition, allow participants to provide suggestions related to them, or even to suggest new ones.
\textbf{Results}: With the results obtained, it was possible to build a list of requirements classified by their priority, along with additional functionalities suggested by the respondents of the questionnaire. Therefore, the development of the tool is already directed and can be started.
\textbf{Preliminary Considerations}: Looking at the hypothesis raised by this study, it is not yet possible to confirm or refute it, as the tool has not yet been developed and tested with end users. However, with the positive results obtained through the survey and the review in the gray literature, it is very likely that it will be confirmed, through the correct and complete development of the final application.

%  Due to the guidelines imposed by Resolution No. 7 of 2018 of the National Education Council (CNE) \cite{Resolucao-MEC:2018}, the curricularization of new outreach actions will become mandatory in 2023. Given that the process of creation and maintenance of an outreach program or project is too time-consuming and added to the imminent obligation, the objective of this work is to implement a tool that supports the process as a whole, allowing from the creation to the issuance of certificates for the participants. For this to be possible, a systematic review was first carried out in the gray literature, in search of similar tools, extracting the most relevant features and aspects among them. Subsequently, using the list reached, a survey was carried out with the possible end users of the academic community of UNIPAMPA, with the objective of classifying the functionalities in order of importance, in addition, allowing the participants to provide suggestions related to them, or even suggesting new ones. The results were analyzed and the production of the proposed solution began, a web tool that will assist in the manual effort required in the processes related to outreach activities. The development of this proposed solution is being carried out by two undergraduate students, dividing the workload involved in the entire Software Engineering process into frontend and backend, this work focuses on the backend.

 \vspace{\onelineskip}
 
 \noindent 
 \textbf{Key-words}: Tool. Survey. Grey Literature. Backend. Outreach. Community. University. NestJS.
\end{resumo}
