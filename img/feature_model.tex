\usetikzlibrary{angles,shadows.blur,positioning,backgrounds}
\forestset{
%   declare count register=disjuncts from,
%   disjuncts from'=0,
%   declare count register=colega from,
  colega colour/.code={\colorlet{colegacol}{#1}},
  autor colour/.code={\colorlet{autorcol}{#1}},
  draw colour/.code={\colorlet{drawcol}{#1}},
%   colega colour=gray,
%   autor colour=white,
  draw colour=black,
  /tikz/colega/.style={fill=white, draw=drawcol},
  /tikz/autor/.style={fill=green!25, draw=drawcol},
  disjunct/.style={
    tikz+={\path (.parent) coordinate (A) -- (!u.children) coordinate (B) -- (!ul.parent) coordinate (C) pic [fill=drawcol] {angle};}
  },
  disjunction tree/.style={
    where={isodd(n_children())}{
      for n={int((n_children()+1)/2)}{calign with current},
    }{
      calign=midpoint,
    },
    before typesetting nodes={
      for nodewalk={
        filter/.wrap pgfmath arg={{level>=##1}{n_children()>1}}{(disjuncts_from)}
      }{
        or,
      },
      tikz+={
        [font=\sffamily]
        \node (l) [anchor=east] at (current bounding box.north west) {Legenda};
        \foreach \i/\j [remember=\i as \k (initially l)] in {autor/Autor,colega/Colega}
        {
          \node (\i) [below=20pt of \k.north, anchor=north, text centered, \i, minimum width=5pt,] {};
          \node (\j) [right=5pt of \i.center, anchor=west] {\j};
        };
        \draw [drawcol] (or.south west) coordinate (A) -- (or.north) coordinate (B) -- (or.south east) coordinate (C) pic [fill=drawcol, angle radius=5pt] {angle};
        \node (c) [below=0pt of Colega.south] {};
        \scoped[on background layer]{\node [draw, fill=white, blur,fit=(l) (Autor) (Colega) (c)] {};}
      },
    },
    for tree={
      parent anchor=children,
      child anchor=parent,
      l'+=10mm,
      fill=gray!15,
      text height=2ex,
      text depth=.5ex,
      font=\sffamily,
    }
  },
}

% \begin{landscape}
\begin{figure}[!htb]
\label{fig:featuremodel}
\resizebox{.8\textwidth}{!}{
\begin{adjustbox}{max width=1.5\textwidth}
%% https://tex.stackexchange.com/questions/335708/feature-diagram-in-latex
    \begin{forest}  
    disjunction tree,
    draw colour=darkgray,
    [Extensionly, fill=blue!15, draw=drawcol
        [Back-end, autor,
            [Manter ACEs, autor]
            [Testes, autor
                [Funcionais, autor]
                [Unitários, autor]
            ]
            [Certificados, autor, name=c
                [Geração, autor]
                [Validação, autor]
                [Solicitação, colega]
            ]
        ]
        [Front-end, colega, name=f 
            [Processos, colega
                [Usuários, colega]
                [Eventos, colega]
                [Atividades, colega]
            ]
            [Divulgação, colega]
        ]
        
    ]
    \draw (f.south) -- (c.north); % certificados
    \end{forest}
    \end{adjustbox}
}
    \centering
    \caption{Separação de tarefas representado por Feature Model. Fonte: Autor}
    \end{figure}
% \end{landscape}

