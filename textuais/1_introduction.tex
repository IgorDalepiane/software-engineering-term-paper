% Introdução - Objetivo fazer um software ..., falar a limitação da unipampa em prover ferramentas (tem um SAP, que é so cadastro de projetos, mas o processo de gestão nao existe, so acompanhamento, nao gera certificado ...) por causa disso muita burocracia, em vez de fazer extensão opta por outra coisa menos burocratica
    % Deiar claro q é extenso, complexo, e que esta fazendo em duas maos, imagem de lego mostrando a coorelação
%==============================================================================
\chapter{INTRODUCTION}\label{introduction}
%==============================================================================
% Unipampa disponibiliza atividades extras, ensino, ...

% Atualmente a UNIPAMPA disponibiliza tres categorias de atividades extras, as de ensino, de pesquisa e de extensão. Atividades de ensino consistem no aprendizado do aluno de modo geral, podem ser cursos, aulas expositivas, atividades de monitoria, entre outras. As atividades de pesquisa são constituidas por tudo que  ́e relacionado a pesquisa em si, entre elas estão as iniciações cientificas, os Trabalhos de Conclusão de Curso (TCC), publicações de trabalhos em eventos, e assim por diante. Por  ́ultimo temos os projetos e programas de extens ̃ao, que s ̃ao o foco deste trabalho, e de acordo com o Plano de Desenvolvimento Institucional (PDI) de 2019, “A extensão assume o papel de promover a relação dialógica com a comunidade externa, pela democratização do acesso ao conhecimento acadêmico bem como pela realimentação das práticas universitárias a partir dessa dinâmica” [UNIPAMPA 2019].

\acl{UNIPAMPA} currently offers three categories of extra activities, teaching, research and outreach. 
Teaching activities consist of student learning in general, they can be courses, lectures, monitoring activities, among others. 
Research activities are constituted by everything that is related to research itself, among them are scientific initiations, \acp{TP}, publication of papers in events, and so on. 
Finally, we have the Outreach Projects and Programs, which are the focus of this work, and according to the 2019 \ac{IDP}, ``Outreach assumes the role of promoting a dialogic relationship with the external community, for the democratization of access to academic knowledge as well as for the feedback of university practices based on this dynamic'' \cite{PDI-Unipampa:2019-2023}.

%===========================================================

% O que é extensão

% Para explicar o que é extensão dentro de um ambiente acadêmico, será utilizado a Resolução Nº 104 de 2015 \cite{Resolucao-104:2015}, que esclarece a extensão como uma ação que incentiva a pesquisa e o desenvolvimento, aumentando o laço entre comunidade e universidade. As atividades extensionistas devem ter obrigatoriamente a participação da comunidade externa e promover o balanco entre atividades práticas e teóricas. Para realizar a classificação entre as atividades, são definidos quatro termos sendo eles: (1) Projetos, "conjunto de ações articuladas em torno de tema e objetivos comuns"; (2) Programas, "conjunto de projetos articulados, podendo contemplar mais de uma modalidade de ação (projeto, cursos, eventos)"; (3) Cursos, "atividades de formação"; (4) Eventos, "atividades de caráter artístico ou científico". Logo é preciso que alguns orgãos fiquem responsáveis pelo gerenciamento destas atividades, também definidos pela Resolução 104, são eles: (1) Pró-Reitoria de Extensão e Cultura (PROEXT); (2) Comissão Superior de Extensão (CSE); (3) Comissão Local de Extensão (CLE).

To explain what outreach is within an academic environment, Resolution No. 332 of 2021 will be used \cite{Resolucao-332:2021}, which clarifies \ac{OA} as an action that encourages research and development, increasing the bond between the community and \ac{HEI}. 
\acp{OA} must have the participation of the external community and promote a balance between practical and theoretical activities. 
To classify these outreach activities, four terms are defined, namely: 
(1) Projects, ``set of actions articulated around a common theme and objectives''; 
(2) Programs, ``set of articulated projects, which may include more than one type of action (project, courses, events)''; 
(3) Courses, ``training activities''; 
(4) Events, ``activities of an artistic or scientific nature''. 
Therefore, it is necessary for some bodies to be responsible for managing these activities, also defined by Resolution 104, they are: (1) \ac{PROEXT}; (2) \ac{CSE}; (3) \ac{CLE}.

%=============================================================

% O que é curricularização

% A curricularização da extensão descrita na Resolução Nº 7 de 2018 \cite{Resolucao-MEC:2018}, explicita que as atividades de extensão devem ter sua proposta, desenvolvimento e conclusão, devidamente registrados, documentados e analisados, de forma que seja possível organizar os planos de trabalho, as metodologias, os instrumentos e os conhecimentos gerados. 
% Também ordenando que as instituições de ensino deveriam incluir em seu Plano de Desenvolvimento Institucional (PDI), no mínimo 10% (dez por cento) da carga horária total do curso voltada para atividades de extensão, além de todos os termos relacionados, com prazo de até três anos, a contar da data de sua homologação. Tendo em vista esta demanda, a UNIPAMPA criou a Resolução \ac{CONSUNI} Nº 317 de 29 de abril de 2021, que implanta todas as diretrizes apresentadas pelo \ac{MEC}.

The curricularization of the outreach described in Resolution No. 7 of 2018 \cite{Resolucao-MEC:2018}, explains that \acp{OA} must have their proposal, development and conclusion, duly recorded, documented and analyzed, so that it is possible to organize work plans, methodologies, instruments and knowledge generated. 
Also ordering that educational institutions should include in their \ac{IDP}, at least 10\% (ten percent) of the total course load focused on \acp{OA}, in addition to all related terms, with a deadline of up to three years from the date of its approval. 
In view of this demand, \acl{UNIPAMPA} created \ac{CONSUNI} Resolution No. 317 of April 29, 2021, \cite{CONSUNI-Unipampa:2021}, which implements all the guidelines presented by the \ac{MEC}.

%==============================================================
% Limitações da Unipampa

% Para controlar tudo isto é indispensável um software completo, e que seja de fácil uso, com que os usuários estejam confortáveis em usar e consigam completar suas tarefas utilizando-o, atualmente a UNIPAMPA apenas possuí um sistema chamado SAP, que serve apenas para o cadastro de projetos de extensão, mas não oferece suporte para a manutenção deles. 
% Por causa disto, acaba fazendo com que a burocracia se concentre fora do sistema, tornando este processo chato e demorado, os professores muitas vezes até desistem de realiza-lo, optando por outras atividades menos burocráticas.

To control all this, a complete software is indispensable, and that is easy to use, with which users are comfortable to use and can complete their tasks using it. 
Currently, \acl{UNIPAMPA} only has a system called \ac{SAP}, which serves only for registration outreach projects, submit proposals to the public notices offered and manage the scholarship holders of the awarded notices, but does not support other processes. 
Because of this, it ends up making the bureaucracy concentrate outside the system, making this process boring and time-consuming, teachers often even give up doing it, opting for other less bureaucratic activities.

%==============================================================
% Unipampa cidadã

% Relacionado a este assunto, pouco tempo atrás foi divulgado a Instrução Normativa Nº18 \cite{unipampacidada}, que estipula as normativas do Programa Institucional ``UNIPAMPA Cidadã''. A qual é um programa de extensão que deverá ser composto por ações de cidadania e solidariedade, como campanha do agasalho, arrecadação de alimentos, suporte a asilos, etc., sendo obrigatório a sua oferta. Quando efetivada, em todos os cursos de graduação, deverá ser alocada uma carga horária mínima de 60 e máxima de 120.

Related to this matter, Normative Instruction No. 18 \cite{unipampacidada} was released a short time ago, which stipulates the norms of the Institutional Program ``UNIPAMPA Cidadã''. 
Which is an outreach program that should be composed of citizenship and solidarity actions, such as clothing campaign, food collection, support for asylums, etc., being mandatory to offer them. 
When effective, in all undergraduate courses, a minimum workload of 60 and a maximum of 120 must be allocated.

%------------------------------------------------------------------------------
\section{Motivation}\label{sec:motivation}
%------------------------------------------------------------------------------
% Trabalho manual -
% Necessidades da comunidade é dificil atender (falta de ferramenta) -
% Opção de professores por fazer algo menos burocratico
% Divulgação espalhada, muitos emails, alunos não veem
% Literatura cinza e nao encontramos a melhor que resolvesse por completo os problemas não apenas parcialmente, citar duas (pode ser as do anteprojeto e mais algumas da literatura) 
%==============================================================
% O processo de curricularização proposto pela Resolução Nº 317 \cite{CONSUNI-Unipampa:2021}, se tornará obrigatório em 2023, tendo em vista o esforço que será necessário para completar demandas manualmente como cadastro, controle, emissão de certificados e ingresso dos participantes, implícitos em uma \ac{OCA}, fora proposto a criação de uma ferramenta de apoio na gestão destes projetos e programas de extensão, conseguindo assim diminuir a burocracia e agilizar o processo.

The process of curricularization of outreach proposed by Resolution Nº 317 \cite{CONSUNI-Unipampa:2021}, will become mandatory in 2023, given the effort that will be required to manually complete demands such as registration, control, issuance of certificates and entry of participants, implicit in an \ac{OCA}, it was proposed to create a support tool in the management of these projects and outreach programs, thus managing to reduce bureaucracy and speed up the process.

%==============================================================

% A comunidade periodicamente entra em contato com a universidade para solicitar algum tipo de ação solidária, com esta demanda são geradas atividades extensionistas, podendo ser exercidas por alunos gerenciados por um professor responsável ou até mesmo dentro de uma disciplina de seus cursos. Mas esta comunicação não é a das mais intuitivas, não possuindo um sistema para gerencialas, leva a unica opção de ter que fazela por meio de ligações ou até mesmo presencialmente, isto é muito desanimador para a comunidade. Tendo em vista isso, uma das motivações do desenvolvimento da ferramenta é fortalecer este laço entre comunicade acadêmica e comunicade externa, permitindo que novas demandas sejam criadas na própria ferramenta.

The community periodically contacts the university to request some type of solidarity action, with this demand, \acp{OA} are generated, which can be carried out by students managed by a coordinator or even within a subject of their courses. 
But this communication is not the most intuitive, not having a system to manage them, it leads to the only option of having to do it through calls or even in person, this is very discouraging for the community. 
In view of this, one of the motivations for the development of the tool is to strengthen this link between academic communication and external communication, allowing new demands to be created in the tool itself.

%==============================================================
% Em relação a divulgação das atividades de extensão, hoje em dia são enviados emails para os alunos informando sobre novas oportunidades, mas geralmente as caixas de entrada dos alunos ja são bombardeadas por emails no dia a dia, levando ao desinteresse de ler todos eles. Por este motivo, com uma ferramenta que concentrasse todas as informações, oportunidades e novidades relacionadas a extensão, os alunos não precisariam mais se aventurar a pesquisar no seu mar de emails quando precisarem procurar por uma nova atividade, apenas recorreriam a ferramenta onde tudo ja esta organizado e pronto para ser utilizado.

Regarding the dissemination of \acp{OA}, nowadays emails are sent to students informing them about new opportunities, but usually students inboxes receives a lot of emails on a daily basis, leading to the lack of interest in reading all of them. 
For this reason, with a tool that concentrated all the information, opportunities and news related to the outreach.
Hence, the students would no longer need to venture into their sea of emails when they need to look for a new activity, they would just resort to the tool where everything is already organized and ready to use.

%==============================================================
% Outro motivador que incentivou o desenvolvimento desta ferramenta é que a partir da revisão na literatura cinza que foi conduzida, não foram encontradas ferramentas que solucionavam por completo os problemas relacionados a estes processos. Algumas ferramentas apresentavam funcionalidades e detalhes que outras não apresentavam e vice-versa, mas que juntas construiriam uma ferramenta íntegra.

Another motivator that encouraged the development of this tool is that from the review in the grey literature that was conducted, no tools were found that completely solved the problems related to these processes. Some tools had features and details that others did not and vice versa, but together they would build a complete tool.

%------------------------------------------------------------------------------
\section{Objectives}\label{sec:objectives}
%------------------------------------------------------------------------------
% Fazer ferramenta que englobe todos os passos
% Deixar mais simples o processo
% Centralização de informações
% Redução do trabalho manual
% Fazer com que a comunidade utilize a ferramenta para sugerir atividades
% Facilitar a comunicação entre aluno e professor
% 

% Tendo em vista o que foi apresentado, o objetivo geral do tema deste TCC, é o desenvolvimento do \textit{back-end} de uma ferramenta que servira de apoio na gestão de programas e projetos de extensão, reproduzindo e auxiliando em todos os processos referentes a este assunto, desde sua criação até a geração de certificados quando for finalizada. O intuito é diminuir o esforço e tempo gasto pelos envolvidos, nestas etapas manuais do processos, além disso, permitir que seja construido um novo canal de comunicação entre comunidade acadêmica e comunidade externa, permitindo sugestões de demandas para atividades de extensão diretamente na ferramenta.

In view of what has been presented, the research aim of the theme of this term paper is the development of the \textit{back-end} of a tool that will serve as support in the management of outreach programs and projects, reproducing and assisting in all processes related to this demand, from its creation to the generation of certificates when it is finalized. 
The aim is to reduce the effort and time spent by those involved in these manual steps of the process. 
In addition to allowing a new communication channel to be built between the academic community and the external community, allowing suggestions for demands for \acp{OA} directly in the tool.

% Portanto, na Tabela \ref{tbl:tableObjectives} é apresentada a síntese dos objetivos gerais e específicos, bem como o assunto, o tema, o problema de pesquisa e a hipótese de solução.

Therefore, the Table \ref{tbl:tableObjectives} presents the synthesis of the research aims and objectives, as well as the subject, the study, the research question (problem) and the solution hypothesis.

\begin{table}[!htb]
  \centering
  \caption{Synthesis of the Research Aim and Research Objectives.}
  \label{tbl:tableObjectives}
  \footnotesize
  \begin{tabular}{l|p{11cm}}
    \bottomrule
    \rowcolor[rgb]{0.749,0.749,0.749} \multicolumn{1}{c|}{\textbf{Topic}}                  & \multicolumn{1}{c}{\textbf{Description}}                                                                                                                                                                                             \\
    \hline
    \rowcolor[rgb]{0.898,0.898,0.898} \textcolor[rgb]{0.145,0.145,0.145}{\textbf{Subject}} & Management of outreach programs and projects.                                                                                                                                                                                        \\
    \textbf{Study}                                                                         & Tool for Support in management of outreach programs and projects.                                                                                                                                                                    \\
    \rowcolor[rgb]{0.898,0.898,0.898} \textbf{Research Question}                           & How can a tool to support the management of outreach programs and projects of \acs{UNIPAMPA} can optimize the management of proposition, registration, dissemination and accountability processes of outreach actions?               \\
    \textcolor[rgb]{0.145,0.145,0.145}{\textbf{Research Hypothesis}}                       & With a tool to support the management of outreach programs and projects, it's possible to have a reduction on the effort needed to create an outreach activity and an increase in the engagement of volunteer outreach participants. \\
    \rowcolor[rgb]{0.898,0.898,0.898} \textbf{Research Aim}                                & Develop the \textit{back-end} of the tool to support the management of outreach programs and projects of \acs{UNIPAMPA}                                                                                                                       \\
    \textbf{Research Objectives}                                                           & Report results and execution methods of the following processes:
    \begin{inparaenum}[(i)]
      \item Research: Analyze similar tools, state the processes that will be made available by the tool, conduct surveys with the organizers and participants of \acp{OA}, understand the limitations of current processes.
      \item Planning: Elicitate functional and non functional requirements, identify stakeholders, define architecture, technologies and tools.
      \item Development: Develop the features raised, build and run test cases.
      \item Deployment: Perform experiments with possible end users, collect feedback and implement appropriate improvements and corrections.
    \end{inparaenum}  
    \\
    \toprule
  \end{tabular}
  \fonte{Author.}
\end{table}

%------------------------------------------------------------------------------
\section{Contribution}\label{sec:contribution}
%------------------------------------------------------------------------------

% Trabalho complexo
% Dois alunos de graduação
% Back e Front, este é back
% Imagem

% A proposta para esta ferramenta é projetada para a participação de dois alunos, Igor Dalepiane da Costa e Lucas Alexandre Fell, pois a sua complexidade é alta, justificando este desenvolvimento em dupla.

The proposal for this tool is designed for the participation of two students, Igor Dalepiane da Costa and Lucas Alexandre Fell, because its complexity is high, justifying this double development.

% Para isto foi feito a divisão entre o desenvolvimento do  \textit{back-end} e  \textit{front-end}, o primeiro sendo desenvolvido por pelo proponente deste trabalho e o segundo pelo outro estudante. Para melhor visualização foi desenvolvido um \textit{feature model} com a divisão exata das tarefas que serão desempenhadas por cada um dos estudantes, representado na Figura 1. São contribuições deste trabalho:

For this, the division was made between the development of \textit{back-end} and \textit{front-end}, the first being developed by the proponent of this work and the second by the other student. For better visualization, a feature model was developed with the exact division of the tasks that will be performed by each of the students, represented in Figure 1. Contributions of this work:

% \begin{itemize}
%     \item Pesquisa entre os docentes da universidade sobre a real necessidade deste instrumento de organização dos processos das \acp{OCA}.
%     \item Desenvolvimento do  \textit{back-end} do sistema, englobando todos os processos que serão disponibilizados pela ferramenta, explicitados na Figura 1.
% \end{itemize}

\begin{itemize}
    \item Research among university professors on the real need for this instrument to organize the \acp{OCA} processes;
    \item Development of the \textit{back-end} of the system, encompassing all the processes that will be made available by the tool, explained in \Cref{fig:featuremodel}.
\end{itemize}

% \usetikzlibrary{angles,shadows.blur,positioning,backgrounds}
\forestset{
%   declare count register=disjuncts from,
%   disjuncts from'=0,
%   declare count register=colega from,
  colega colour/.code={\colorlet{colegacol}{#1}},
  autor colour/.code={\colorlet{autorcol}{#1}},
  draw colour/.code={\colorlet{drawcol}{#1}},
%   colega colour=gray,
%   autor colour=white,
  draw colour=black,
  /tikz/colega/.style={fill=white, draw=drawcol},
  /tikz/autor/.style={fill=green!25, draw=drawcol},
  disjunct/.style={
    tikz+={\path (.parent) coordinate (A) -- (!u.children) coordinate (B) -- (!ul.parent) coordinate (C) pic [fill=drawcol] {angle};}
  },
  disjunction tree/.style={
    where={isodd(n_children())}{
      for n={int((n_children()+1)/2)}{calign with current},
    }{
      calign=midpoint,
    },
    before typesetting nodes={
      for nodewalk={
        filter/.wrap pgfmath arg={{level>=##1}{n_children()>1}}{(disjuncts_from)}
      }{
        or,
      },
      tikz+={
        [font=\sffamily]
        \node (l) [anchor=east] at (current bounding box.north west) {Legenda};
        \foreach \i/\j [remember=\i as \k (initially l)] in {autor/Autor,colega/Colega}
        {
          \node (\i) [below=20pt of \k.north, anchor=north, text centered, \i, minimum width=5pt,] {};
          \node (\j) [right=5pt of \i.center, anchor=west] {\j};
        };
        \draw [drawcol] (or.south west) coordinate (A) -- (or.north) coordinate (B) -- (or.south east) coordinate (C) pic [fill=drawcol, angle radius=5pt] {angle};
        \node (c) [below=0pt of Colega.south] {};
        \scoped[on background layer]{\node [draw, fill=white, blur,fit=(l) (Autor) (Colega) (c)] {};}
      },
    },
    for tree={
      parent anchor=children,
      child anchor=parent,
      l'+=10mm,
      fill=gray!15,
      text height=2ex,
      text depth=.5ex,
      font=\sffamily,
    }
  },
}

\begin{figure}[!htb]
\label{fig:featuremodel}
\begin{forest} 
    disjunction tree,
    draw colour=darkgray,
    [Extensionly, fill=blue!15, draw=drawcol
        [Back-end, autor,
            [Manter ACEs, autor]
            [Testes, autor
                [Funcionais, autor]
                [Unitários, autor]
            ]
            [Certificados, autor, name=c
                [Geração, autor]
                [Validação, autor]
                [Solicitação, colega]
            ]
        ]
    ]
\end{forest}
\end{figure}

%------------------------------------------------------------------------------
\section{Organization}\label{sec:organization}
%------------------------------------------------------------------------------

This document is organized according to the following:

\begin{itemize}
    \item  \textbf{Chapter 2: Methodology:} Details of the the methodology adopted during the search, along with it's classification and research schedule.
    \item  \textbf{Chapter 3: Background:} Details of main concepts related to this work, such as, resolutions and \acp{OA}.
    \item  \textbf{Chapter 4: Grey Literature:} This chapter presents in more detail the review performed in the grey literature to find similar tools.
    \item  \textbf{Chapter 5: Survey:} Provides details about the survey performed, its protocol and results.
    \item  \textbf{Chapter 6: Extensionly:} Provides details of the design and implementation
of the proposed tool.
    \item  \textbf{Chapter 7: Conclusions:} This chapter presents the partial conclusions about this study.
\end{itemize}



