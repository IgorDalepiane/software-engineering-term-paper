%==============================================================================
\chapter{PRELIMINARY CONCLUSIONS}\label{conclusions}
%==============================================================================
% 1-Recapitular o tema do trabalho, aspectos da justificativa, motivos de escolha do tema, contribuições do trabalho
% 1-Metodologia do trabalho
% Esse trabalho pretendeu entender [apresente o tema] para [justificativa do trabalho], a partir de [metodologia utilizada].
%==============================================================================
% O presente trabalho buscou se aprofundar em processos de gerenciamento de atividades de extensão, para construir uma ferramenta que de suporte nos processos burocraticos envolvidos, facilite a comunicação com a comunidade externa, e aumente a divulgação e integração dos alunos com as atividades. Para isso utilizou-se de duas metodologias, uma revisão na literatura cinza para encontrar soluções similares ja existentes elicitando as suas funcionalidades principais, e um survey com os possíveis usuários finais da ferramenta, em busca de ordenar os requisitos baseado em sua importância, também adiquirindo mais sugestões relacionadas ao assunto.

The present work sought to delve into processes of management of \aclp{OA}, to build a tool that supports the bureaucratic processes involved, facilitates communication with the external community, and increases the dissemination and integration of students with the activities. 
For this, two methodologies were used, a review in the grey literature to find similar solutions that already exist, eliciting their main functionalities, and a survey with the possible end users of the tool, in search of ordering the requirements based on their importance, also acquiring more suggestions related to the subject.
%==============================================================================

%==============================================================================
% 2-Objetivos do TCC, atingiu os objetivos? explicar a opinião
% Para se atingir uma compreensão do [objetivo geral], definiu-se três objetivos específicos. O primeiro [objetivo específico]. Verificou-se que [resultado]. Depois, [objetivo específico 2]. A análise permitiu concluir que [resultado].
%==============================================================================
%==============================================================================
% 3 - apresentar resultados obtidos(sem dados), não pode aparecer nenhuma informação NOVA, 
% 3 - Avanços em relação aos objetivos definidos
%==============================================================================
% Para ser possivel atingir o objetivo geral do trabalho, que é Develop the backend of the tool to support the management of outreach programs and projects of \ac{UNIPAMPA}, foram definidos cinco objetivos especificos. 
% O primeiro objetivo era de realizar a revisão na literatura cinza em busca de funcionalidades em ferramentas similares, com os resultados obtidos pode-se verificar que este foi atingido, pois foi possível levantar diversas ferramentas e no final sair com uma lista de tamanho significante juntamente com suas funcionalidades e detalhes especificos, contribuindo muito no planejamento.  
% O segundo objetivo esta relacionado com a elaboração de um survey para compreender as opiniões dos possíveis usuários finais, este objetivo foi atingido com sucesso pois foi possivel analisar as dores e preocupações dos usuários, sendo possível com os resultados, extrair várias melhorias e funcionalidades para a ferramenta proposta.

In order to achieve the general objective of the work, which is develop the backend of the tool to support the management of outreach programs and projects of \ac{UNIPAMPA}, five specific objectives were defined.
The first objective was to carry out a review in the grey literature in search of features in similar tools, with the results obtained it can be seen that this was achieved, as it was possible to raise several tools and in the end to come out with a list of significant size together with its functionalities and specific details, contributing a lot in planning.
The second objective is related to the elaboration of a survey to understand the opinions of the possible end users, this objective was successfully achieved because it was possible to analyze the pains and concerns of the users, being possible with the results, to extract several improvements and functionalities for the proposed tool.

%==============================================================================
% A construção de um roadmap e tasks concretas para o desenvolvimento, define o terceiro objetivo especifico, este foi atingido de modo parcial pois a transformação de \ac{FR} para tarefas de desenvolvimento apenas ocorrerá no desenvolvimento da segunda parte deste trabalho, durante o \ac{TP} II. 
% O quarto objetivo diz sobre o estudo, analise e escolha de uma stack de desenvolvimento para o backend da ferramenta proposta, incluindo linguagem de programação, arquiteturas e frameworks, este objetivo foi atingido com sucesso, as escolhas feitas podem ser vistas na \Cref{extensionly}. 
% Como quinto e ultimo objetivo especifico está no próprio desenvolvimento de um \ac{MVP} do backend da ferramenta, este objetivo ainda não foi atingido, pois está planejado que o desenvolvimento venha a acontecer ainda.

The construction of a roadmap and concrete tasks for the development, defines the third specific objective, this was partially achieved because the transformation from \ac{FR} to development tasks will only occur in the development of the second part of this work, during the \ac{TP} II.
The fourth objective says about the study, analysis and choice of a development stack for the backend of the proposed tool, including programming language, architectures and frameworks, this objective was successfully achieved, the choices made can be seen in \Cref{extensionly}.
As the fifth and last specific objective is the development of an \ac{MVP} for the tool's backend, this objective has not yet been achieved, as it is planned that the development will happen yet.
%==============================================================================
% 4 - Verificar a hipotese, confirmou ou refutou? explicar
% Com isso, a hipótese do trabalho de que [hipótese] se [ confirmou ou se refutou], por [motivos].
%==============================================================================
% Com os objetivos e resultados analisados, a hipotese deste trabalho de que ``With a tool to support the management of outreach programs and projects, it’s possible to have a reduction on the effort needed to create an outreach activity and an increase in the engagement of volunteer outreach participants'', ainda não pode ser confirmada ou refutada, pelo fato de que ainda não foi desenvolvida a aplicação e testada com os usuários finais. Mas tendo em vista todas as reações positivas que os respondentes proporcionaram por meio do survey, é bem provável que com o desenvolvimento correto e completo da aplicação, esta hipotese seja confirmada.

With the objectives and results analyzed, the hypothesis of this work is that ``With a tool to support the management of outreach programs and projects, it's possible to have a reduction on the effort needed to create an outreach activity and an increase in the engagement of volunteer outreach participants'', cannot yet be confirmed or refuted, as the application has not yet been developed and tested with end users. 
But in view of all the positive reactions that the respondents provided through the survey, it is very likely that with the correct and complete development of the application, this hypothesis will be confirmed.

%==============================================================================
% 5- Responder a questão de pesquisa, apresentar a resposta que foi feita depois da pesquisa
% Sendo assim, [resposta ao problema de pesquisa].
%==============================================================================
% Sendo assim, respondendo a questão de pesquisa deste trabalho, presenta na \Cref{tbl:tableObjectives}, é possível com uma ferramenta deste tipo, diminuir o trabalho manual e repetitivo necessário no registro de novas propostas de ações de extensão, facilitando processos como por exemplo geração de certificados de maneira mais eficiente, também com uma ferramenta que centraliza as informações relacionadas a extensão, os alunos saberão onde procurar quando precisarem, aumentando a disseminação de novas ações. Com a ferramenta a relação entre professor e participante, também sera fortalecida, permitindo experiências e interações mais significativas entre eles.

Therefore, answering the research question of this work, presented in \Cref{tbl:tableObjectives}, it is possible with a tool of this type, to reduce the manual and repetitive work necessary in the registration of new proposals for outreach actions, facilitating processes such as generate certificates more efficiently, also with a tool that centralizes information related to the outreach, students will know where to look when they need to, increasing the dissemination of new actions. 
With the tool, the relationship between teacher and participant will also be strengthened, allowing for more meaningful experiences and interactions between them.

%==============================================================================
% 6 - Avaliar a participação do Survey para obter as respostas da pesquisa
% Os instrumentos de coleta dos danos permitiram [avaliação dos instrumentos].
%==============================================================================
% Os resultados obtidos com o survey permitiram os pesquisadores entender a opinião que os possíveis usuários finais tem em relação a existir uma ferramenta como essa para auxiliar-los em processos do genero, com ele foi possível reavaliar algumas questões de implementação, levando em conta sugestões levantadas. É seguro dizer que sem uma pesquisa de opinião com estes usuários, a ferramenta correria muito perigo de não contemplar as expectativas previamente definidas.

The results obtained with the survey allowed the researchers to understand the opinion that possible end users have in relation to the existence of a tool like this to assist them in similar processes, with it it was possible to reassess some implementation issues, taking into account suggestions raised. 
It is safe to say that without an opinion survey with these users, the tool would be in great danger of not meeting the expectations previously defined.

%==============================================================================
%  7 - Propor melhoirias e trabalhos futuros
% Em pesquisas futuras, pode-se [melhorias e direcionamentos].
%==============================================================================
% Para a segunda versão deste \ac{TP}, pretende-se focar muito tempo no desenvolvimento da aplicação, para que seja possível executar um teste real com docentes e discentes, utilizando as funcionalidades presentes na ferramenta com programas e projetos de extensão reais da \ac{UNIPAMPA}, conseguindo assim feedback de dentro da ferramenta, possivelmente realizando outro survey para a coleta de dados.

For the second version of this \ac{TP}, we intend to focus a lot of time on the development of the application, so that it is possible to run a real test with teachers and students, using the features present in the tool with real outreach programs and projects from \ac{UNIPAMPA}, thus obtaining feedback from within the tool, possibly carrying out another survey for data collection.
%==============================================================================
