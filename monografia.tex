%% monografia.tex, fabiokepler, jeancheiran
%% Copyright 2012-2018 by UNIPAMPA LaTeX group at https://bitbucket.org/unipampaalegrete/monografias-cc-es-repo/
%%
%% This work may be distributed and/or modified under the conditions of the LaTeX Project Public
%% License, either version 1.3 of this license or (at your option) any later version.
%% The latest version of this license is in
%%   http://www.latex-project.org/lppl.txt
%% and version 1.3 or later is part of all distributions of LaTeX version 2005/12/01 or later.
%%
%% Based on the example file abtex2-modelo-trabalho-academico.tex of the abntex2 package
%% (http://abntex2.googlecode.com/) and on the ppgccufmg 1.45beta2 class
%% (http://vilarneto.com/ppgccufmg,
%% http://www.dcc.ufmg.br/pos/alunos/modelodisstese.php
%% and http://www.dcc.ufmg.br/~mirella).
%%
%% Adapted for the Computer Science program at UNIPAMPA (http://www.unipampa.edu.br)
%% by Fabio Kepler (fabio@kepler.pro.br) and Jean Cheiran (jeancheiran@unipampa.edu.br).
%%
%% Version 2.5 - 2018/08
%% Version 2.4 - 2017/05
%% Version 2.3 - 2013/03

% +++++++++++++++++++++++++++++++++++++++++++++++++++++++++++++++++++++++++++++++++++++++++++++++++
% Este modelo utiliza o pacote abnTeX2. Veja como instalá-lo em seu ambiente em
% http://abntex2.googlecode.com/.
% -------------------------------------------------------------------------------------------------
% abnTeX2: Modelo de Trabalho Acadêmico (tese de doutorado, dissertação de
% mestrado e trabalhos monográficos em geral) em conformidade com
% ABNT NBR 14724:2011: Informação e documentação - Trabalhos acadêmicos -
% Apresentação
% -------------------------------------------------------------------------------------------------
% Normas institucionais utilizadas:
% http://porteiras.r.unipampa.edu.br/portais/sisbi/programa-de-capacitacao/
% +++++++++++++++++++++++++++++++++++++++++++++++++++++++++++++++++++++++++++++++++++++++++++++++++

\documentclass[12pt,openright,twoside,a4paper,chapter=TITLE,english]{abntex2}    % frente e verso
%\documentclass[12pt,oneside,a4paper]{abntex2}            % apenas frente

% +++++++++++++++++++++++++++++++++++++++++++++++++++++++++++++++++++++++++++++++++++++++++++++++++
% PACOTES
% -------------------------------------------------------------------------------------------------
% Pacotes fundamentais
\usepackage[style=abnt,backref=true]{biblatex}
\usepackage{cmap}           % Mapeamento de caracteres especiais no PDF
\usepackage{lmodern}        % Usa fonte Latin Modern
\usepackage[T1]{fontenc}    % Seleção de codificação de fonte
\usepackage[utf8]{inputenc} % Codificação do arquivo (conversão automática dos acentos)
\usepackage[brazil]{babel}  % Idioma para hifenização e tradução de vários elementos
\usepackage{makeidx}        % Criação de índice
\usepackage{hyperref}       % Formatação do índice
\usepackage{lastpage}       % Usado pela Ficha catalográfica
\usepackage{indentfirst}    % Indenta o primeiro parágrafo de cada seção
\usepackage[table,xcdraw,usenames,dvipsnames]{xcolor}  % Controle das cores (com nomes) e formatação de tabelas
\usepackage{graphicx}       % Inclusão de gráficos
\usepackage{forest}
\usepackage{booktabs}       % Formatação de tabelas
\usepackage{adjustbox}

% \usepackage{rotating}
\usepackage{lscape}

% -------------------------------------------------------------------------------------------------
% Para citações
% \usepackage[brazilian,hyperpageref]{backref} % Páginas com as citações na bibliografia
% \usepackage[alf,abnt-emphasize=bf]{abntex2cite} % Citações padrão ABNT (alfanumérico)

% -------------------------------------------------------------------------------------------------
% Pacotes opcionais
\usepackage{nomencl}        % Para criar uma lista de símbolos
\usepackage{acro}           % Para usar acrônimos e abreviaturas
\usepackage{tikz}           % Para fazer figuras, diagramas e gráficos integrados e elegantes
\usepackage{pgfplots}       % Usa o pacote tikz para fazer gráficos muito melhores que os do Excel
% \usepackage{pgfplotstable}  % Para gerar tabelas automaticamente a partir de arquivos com dados
\usepackage{filecontents}   % Para colocar o conteúdo de um arquivo dentro de um arquivo tex
\usepackage{todonotes}      % Para criar anotações durante o desenvolvimento do texto
\usepackage{multirow}       % Permite fazer tabelas com múltiplas linhas
%\let\newfloat=\undefined    % Workaround para usar o pacote algorithm
%\usepackage{algorithm}      % Para escrever algoritmos
%\usepackage{clrscode}       % Para escrever algoritmos
%\usepackage{clrscode3e}     % Para escrever algoritmos; mais simples que os pacotes acima
\usepackage{pdfpages}        % Para incluir a folha de aprovação assinada em PDF
\usepackage[olditem]{paralist}  % Listas in-line
\usepackage{colortbl}        % Cronograma (Tabelas)
\usepackage[autostyle]{csquotes} % Citações
\usepackage{url}
\usepackage{hhline}
\usepackage[nameinlink]{cleveref}
\usepackage{makecell}
\usepackage{rotating}

% \usepackage{enumerate}
\usepackage{enumitem}

\usetikzlibrary{patterns}

\usepackage{smartdiagram}

\usepackage{amssymb}

\usepackage{pgf-pie}

\smartdiagramset{circular distance=3.5cm, font=\medium, text width=2.0cm,
module minimum width=1.5cm, module minimum height=1.0cm,arrow tip=to,uniform connection color=true,uniform color list=blue!30 for 7 items}

\def\BibTeX{{\rm B\kern-.05em{\sc i\kern-.025em b}\kern-.08em
    T\kern-.1667em\lower.7ex\hbox{E}\kern-.125emX}}

%PGPLOTS
\usepackage{pgfplots}
\usetikzlibrary{patterns}
\usepgfplotslibrary[statistics] % ConTEXt
\usetikzlibrary{pgfplots.statistics} % LATEX and plain TEX
\usepackage{tikz}
\usepackage{ragged2e}

\usetikzlibrary{patterns}
%\rowcolors{1}{gray!10}{}

\usetikzlibrary{angles,shadows,positioning,backgrounds,mindmap}

\pgfplotsset{width=9cm, height=8cm, compat=1.9}

\usetikzlibrary{shapes,backgrounds}
\def\BibTeX{{\rm B\kern-.05em{\sc i\kern-.025em b}\kern-.08em
    T\kern-.1667em\lower.7ex\hbox{E}\kern-.125emX}}

% começo da bagunça
\usetikzlibrary{angles,shadows.blur,positioning,backgrounds}
\usepackage{forest}
\forestset{
  declare count register=disjuncts from,
  disjuncts from'=0,
  declare count register=concrete from,
  concrete from'=2,
  concrete colour/.code={\colorlet{concretecol}{#1}},
  abstract colour/.code={\colorlet{abstractcol}{#1}},
  draw colour/.code={\colorlet{drawcol}{#1}},
  concrete colour=gray,
  abstract colour=white,
  draw colour=black,
  /tikz/mandatory/.style={circle, fill=drawcol, draw=drawcol},
  /tikz/optional/.style={circle, draw=drawcol, fill=white},
  /tikz/concrete/.style={fill=concretecol, draw=drawcol},
  /tikz/abstract/.style={fill=abstractcol, draw=drawcol},
  /tikz/or/.style={},
  mandatory/.style={edge label={node [mandatory] {}}},
  optional/.style={edge label={node [optional] {}}},
  or/.style={for first={disjunct}},
  disjunct/.style={
    tikz+={\path (.parent) coordinate (A) -- (!u.children) coordinate (B) -- (!ul.parent) coordinate (C) pic [fill=drawcol] {angle};}
  },
  disjunction tree/.style={
    where={isodd(n_children())}{
      for n={int((n_children()+1)/2)}{calign with current},
    }{
      calign=midpoint,
    },
    before typesetting nodes={
      for nodewalk={
        filter/.wrap pgfmath arg={{level>=##1}{n_children()>1}}{(disjuncts_from)}
      }{
        or,
      },
      where={level()>=(concrete_from)}{
        concrete,
      }{
        abstract,
      },
      tikz+={
        [font=\sffamily]
        \node (l) [anchor=north east, xshift=10pt] at (current bounding box.north east) {Legend};
        \foreach \i/\j [remember=\i as \k (initially l)] in {mandatory/Mandatory,optional/Optional,or/Or,abstract/Abstract,concrete/Concrete}
        {
          \node (\i) [below=20pt of \k.north, anchor=north, text centered, \i, minimum width=5pt,] {};
          \node (\j) [right=15pt of \i.center -| mandatory.west, anchor=west] {\j};
        };
        \draw [drawcol] (or.south west) coordinate (A) -- (or.north) coordinate (B) -- (or.south east) coordinate (C) pic [fill=drawcol, angle radius=5pt] {angle};
        \foreach \i in {mandatory,optional} \draw [darkgray] (\i.north east) -- +(45:5pt);
        \node (c) [below=0pt of Concrete.south] {};
        \scoped[on background layer]{\node [draw, fill=white, blur shadow, fit=(l) (Mandatory) (Optional) (Or) (Abstract) (Concrete) (c), rounded corners] {};}
      },
    },
    for tree={
      parent anchor=children,
      child anchor=parent,
      l'+=10mm,
      blur shadow,
      rounded corners,
      text height=2ex,
      text depth=.5ex,
      font=\sffamily,
    }
    % qtree/.style={
    %     for tree={
    %         parent anchor=south, 
    %         child anchor=north,
    %         align=center,
    %         inner sep=0pt,
    %         l'+=10mm,
    %         blur shadow,
    %         rounded corners,
    %         text height=2ex,
    %         text depth=.5ex,
    %         font=\sffamily,
    %     }
    % },
  },
}

% -------------------------------------------------------------------------------------------------
% Configurações de pacotes
% -------------------------------------------------------------------------------------------------

\addto\captionsenglish{% ingles
  %% adjusts names from abnTeX2
  \renewcommand{\folhaderostoname}{Title page}
  \renewcommand{\epigraphname}{Epigraph}
  \renewcommand{\dedicatorianame}{Dedication}
  \renewcommand{\errataname}{Errata sheet}
  \renewcommand{\agradecimentosname}{Acknowledgements}
  \renewcommand{\anexoname}{ANNEX}
  \renewcommand{\anexosname}{Annex}
  \renewcommand{\apendicename}{APPENDIX}
  \renewcommand{\apendicesname}{Appendix}
  \renewcommand{\orientadorname}{Supervisor:}
  \renewcommand{\coorientadorname}{Co-supervisor:}
  \renewcommand{\folhadeaprovacaoname}{Approval}
  \renewcommand{\resumoname}{Resumo} 
  \renewcommand{\listadesiglasname}{List of Abbreviations and acronyms}
  \renewcommand{\listadesimbolosname}{List of Symbols}
  \renewcommand{\fontename}{Source}
  \renewcommand{\notaname}{Note}
   %% adjusts names used by \autoref
  \renewcommand{\pageautorefname}{page}
  \renewcommand{\chapterautorefname}{Chapter}
  \renewcommand{\figureautorefname}{Figure}
  \renewcommand{\tableautorefname}{Table}
  \renewcommand{\sectionautorefname}{Section}
  \renewcommand{\subsectionautorefname}{Section}
  \renewcommand{\subsubsectionautorefname}{Section}
  \renewcommand{\paragraphautorefname}{Section}
%   \renewcommand{\englishbibname}{References}
%   \renewcommand{\englishindexname}{Index}
%   \renewcommand{\englishlistfigurename}{List of Figures}
%   \renewcommand{\englishfigurename}{Figure}
%   \renewcommand{\englishlisttablename}{List of Tables}
%   \renewcommand{\englishtablename}{Table}
%   \renewcommand{\englishcontentsname}{List of Contents}
%   \renewcommand{\englishchaptername}{Chapter}
  \renewcommand{\imprimirtipotrabalho}{Term Paper }
}

% -------------------------------------------------------------------------------------------------
% Configurações do pacote backref
% Usado sem a opção hyperpageref de backref
% \renewcommand{\backrefpagesname}{Cited in page(s):~}
% \renewcommand*{\backrefsep}{,~}%
% \renewcommand*{\backreftwosep}{,~}% ', and~'
% \renewcommand*{\backreflastsep}{,~}% ' and~'
% % Texto padrão antes do número das páginas
% \renewcommand{\backref}{}
% % Define os textos da citação
% \renewcommand*{\backrefalt}[4]{
%     \ifcase #1 %
%         No text citation.%
%     \or
%         Cited in page #2.%
%     \else
%         Cited #1 times on pages #2.%
%     \fi}%
% -------------------------------------------------------------------------------------------------
% Configurações de aparência do PDF final
%\definecolor{blue}{RGB}{41,5,195}
% \definecolor{webgreen}{rgb}{0,.5,0}
% Metainformações do PDF e cores dos links
\hypersetup{
  portuguese,
  %backref=true,
  %pagebackref=true,
  %bookmarks=true,             % show bookmarks bar?
  %bookmarksnumbered=true,
  bookmarksdepth=4,
  pdftitle={\@title},
  pdfauthor={\@author},
  pdfsubject={\imprimirpreambulo},
  pdfkeywords={UNIPAMPA}{Computação}{UNIPAMPA}{abntex}{TCC},
  %pdfproducer={LaTeX with abnTeX2},     % producer of the document
  pdfcreator={\@author},
  colorlinks=true,           % false: boxed links; true: colored links
  linkcolor=black,            % color of internal links
  citecolor=black,            % color of links to bibliography
  filecolor=black,         % color of file links
  urlcolor=black
}
%   linktocpage,
%   colorlinks,
%   citecolor=webgreen,
%   urlcolor=Maroon,
%   linkcolor=RoyalBlue,
%   filecolor=black,
% -------------------------------------------------------------------------------------------------
% Espaçamentos entre linhas e parágrafos
% O tamanho do parágrafo é dado por
\setlength{\parindent}{1.3cm}
% Controle do espaçamento entre um parágrafo e outro
\setlength{\parskip}{0.2cm} % tente também \onelineskip
% Controles do espaçamento entre linhas
%\OnehalfSpacing       % espaçamento um e meio (padrão);
%\DoubleSpacing        % espaçamento duplo
%\SingleSpacing        % espaçamento simples
% -------------------------------------------------------------------------------------------------
% Para o pacote de acrônimos
\acsetup{make-links} %first-style=short}
% -------------------------------------------------------------------------------------------------
% Para o pacote tikz, pgfplots e pgfplotstable
\usetikzlibrary{arrows,chains,matrix,positioning,decorations.pathreplacing,calc}
% -------------------------------------------------------------------------------------------------
% Para poder usar subfiguras e subtabelas
\newsubfloat{figure}
\newsubfloat{table}
\providecommand*{\subfigureautorefname}{\figureautorefname}
% +++++++++++++++++++++++++++++++++++++++++++++++++++++++++++++++++++++++++++++++++++++++++++++++++


% +++++++++++++++++++++++++++++++++++++++++++++++++++++++++++++++++++++++++++++++++++++++++++++++++
% Informações de dados para CAPA e FOLHA DE ROSTO
% -------------------------------------------------------------------------------------------------
\titulo{Extensionly - A tool for supporting the management of outreach projects and programs in the university: Backend}

\autor{Igor Dalepiane da Costa}
\local{Alegrete}
\data{2022}
\orientador{Prof. PhD. Maicon Bernardino da Silveira}

% \coorientador{Prof. <titulação> Nome do Coorientador} % Se houver
\instituicao{FEDERAL UNIVERSITY OF PAMPA}
\tipotrabalho{Projeto de Trabalho de Conclusão de Curso~} % Para TCC I
% \tipotrabalho{Trabalho de Conclusão de Curso~} % Para TCC II
% O preambulo deve conter o tipo do trabalho, o objetivo, o nome da instituição e a área de concentração
\preambulo{\imprimirtipotrabalho presented in Software Engineering Graduation Course in the Federal University of Pampa as a partial requirement for obtaining the title of Software Engineering
Bachelor}
% +++++++++++++++++++++++++++++++++++++++++++++++++++++++++++++++++++++++++++++++++++++++++++++++++
\setlength\bibitemsep{1.8\itemsep}
\addbibresource{bibliografia.bib}
\makeindex % Compila o indice
\makenomenclature % Compila a lista de abreviaturas e siglas

% Abreviaturas 
\DeclareAcronym{fig}{
  short = Fig.,
  long  = Figura,
  tag = abreviaturas
}
% Acrônimos/Siglas
\DeclareAcronym{API}{
  short = API,
  long  = Application Programming Interface,
  tag = acronimos
}
\DeclareAcronym{ATE}{
  short = ATE,
  long  = Administrative Technician in Education,
  tag = acronimos
}
\DeclareAcronym{CAEX}{
  short = CAEX,
  long  = Outreach Actions Control,
  tag = acronimos
}
\DeclareAcronym{CLE}{
  short = CLE,
  long  = Local Outreach Committee,
  tag = acronimos
}
\DeclareAcronym{CONSUNI}{
  short = CONSUNI,
  long  = University Council,
  tag = acronimos
}
\DeclareAcronym{CSE}{
  short = CSE,
  long  = Superior Outreach Committee,
  tag = acronimos
}
\DeclareAcronym{DBMS}{
  short = DBMS,
  long  = Database Management System,
  tag = acronimos
}
\DeclareAcronym{FOREXT}{
  short = FOREXT,
  long  = National Forum for Outreach and Community Action of Universities and Community Higher Education Institutions,
  tag = acronimos
}
\DeclareAcronym{FORPROEX}{
  short = FORPROEX,
  long  = Forum of Pro-Rectors for Outreach of Brazilian Public Universities,
  tag = acronimos
}
\DeclareAcronym{FR}{
  short = FR,
  long  = Functional Requirement,
  tag = acronimos
}
\DeclareAcronym{HEI}{
  short = HEI,
  long  = Higher Education Institution,
  tag = acronimos
}
\DeclareAcronym{HTTP}{
  short = HTTP,
  long  = Hypertext Transfer Protocol,
  tag = acronimos
}
\DeclareAcronym{HECI}{
  short = HECI,
  long  = Higher Education Community Institution,
  tag = acronimos
}
\DeclareAcronym{ID}{
  short = ID,
  long  = {Identification},
  tag = acronimos
}
\DeclareAcronym{IDP}{
  short = IDP,
  long  = Institutional Development Plan,
  tag = acronimos
}
\DeclareAcronym{JS}{
  short = JS,
  long  = JavaScript,
  tag = acronimos
}
\DeclareAcronym{MEC}{
  short = MEC,
  long  = Ministry of Education,
  tag = acronimos
}
\DeclareAcronym{MoSCoW}{
  short = MoSCoW,
  long  = {Must have, Should have, Could have and Will not have},
  tag = acronimos
}
\DeclareAcronym{MVP}{
  short = MVP,
  long  = Minimum Viable Product,
  tag = acronimos
}
\DeclareAcronym{MySQL}{
  short = MySQL,
  long  = My Structured Query Language,
  tag = acronimos
}
\DeclareAcronym{NGO}{
  short = NGO,
  long  = Non-Governmental Organization,
  long-plural-form = Non-Governmental Organizations,
  tag = acronimos
}
\DeclareAcronym{OA}{
  short = OA,
  long  = Outreach Activity,
  long-plural-form = Outreach Activities,
  tag = acronimos
}
\DeclareAcronym{ORM}{
  short = ORM,
  long  = Object Relational Mapper,
  long-plural-form = Object Relational Mappers,
  tag = acronimos
}

\DeclareAcronym{PROEXT}{
  short = PROEXT,
  long  = Dean of Outreach and Culture,
  tag = acronimos
}
\DeclareAcronym{OCA}{
  short = OCA,
  long  = Outreach Curriculum Activity,
  long-plural-form = Outreach Curriculum Activities,
  tag = acronimos
}
\DeclareAcronym{ProExt}{
  short = ProExt,
  long  = University Outreach Program,
  tag = acronimos
}
\DeclareAcronym{PR}{
  short = PR,
  long  = Pull Request,
  tag = acronimos
}
\DeclareAcronym{REST}{
  short = REST,
  long  = Representational State Transfer,
  tag = acronimos
}
\DeclareAcronym{SAP}{
  short = SAP,
  long  = Academic Project System,
  tag = acronimos
}
\DeclareAcronym{SEI}{
  short = SEI,
  long  = Electronic Information System,
  tag = acronimos
}
\DeclareAcronym{SGCE}{
  short = SGCE,
  long  = Electronic Certificate Management System,
  tag = acronimos
}
\DeclareAcronym{SIGAA}{
  short = SIGAA,
  long  = Integrated Academic Activities Management System,
  tag = acronimos
}
\DeclareAcronym{SIPPEE}{
  short = SIPPEE,
  long  = {Information System for Research, Teaching and Outreach Projects},
  tag = acronimos
}
\DeclareAcronym{TP}{
  short = TP,
  long  = Term Paper,
  long-plural-form = Term Papers,
  tag = acronimos
}
\DeclareAcronym{TS}{
  short = TS,
  long  = TypeScript,
  tag = acronimos
}
\DeclareAcronym{UNIPAMPA}{
  short = UNIPAMPA,
  long  = Federal University of Pampa,
  tag = acronimos
}
\DeclareAcronym{CD}{
  short = CD,
  long  = Continuous Deployment,
  tag = acronimos
}
\DeclareAcronym{CDE}{
  short = CDE,
  long  = Continuous Delivery,
  tag = acronimos
}
\DeclareAcronym{CI}{
  short = CI,
  long  = Continuous Integration,
  tag = acronimos
}
\DeclareAcronym{DevOps}{
  short = DevOps,
  long  = Development Operations,
  tag = acronimos
}
\DeclareAcronym{PaaS}{
  short = PaaS,
  long  = Platform as a Service,
  tag = acronimos
}





















% \DeclareAcronym{FORPROEX}{
%   short = FORPROEX,
%   long  = Forum of Pro-Rectors for Outreach of Brazilian Public Universities,
%   tag = acronimos
% }





% \DeclareUnicodeCharacter{0301}{*************************************}
% ************************************************************************************************
\begin{document}
\include{consertos} % Inclui alguns ajustes finos para que fique de acordo com o Manual de Normatização
\selectlanguage{english}

\imprimircapa % Capa [OBRIGATÓRIO]
\imprimirfolhaderosto % Folha de rosto [OBRIGATÓRIO]

\input{pretextuais/folhadeaprovacao} % Folha de aprovação [OBRIGATÓRIO]
\begin{dedicatoria}
   \vspace*{\fill}
   \begin{flushright}
     I dedicate this work to my family and to God, \\
     who have always been my greatest strengths.
   \end{flushright}
   \vspace*{\fill}
\end{dedicatoria}
 % Dedicatória [OPCIONAL]
\begin{agradecimentos}

First and foremost, I would like to thank my family Eliane, Jair and Mateus, who always helped me in all my obstacles and above all taught me the values, love and religiosity that I carry with me to this day. I also thank all the family members who gave me strength and support to keep me on the road.

A special thanks to my advisor Maicon Bernardino who has always been there to teach and guide me through this journey.

I also thank my colleague Lucas Fell who has been with me since the beginning of college, helping to overcome the challenges along the way.

Thank you for everything and God bless you all!

\end{agradecimentos} % Agradecimentos [OPCIONAL]
\begin{epigrafe}
  \vspace*{\fill}
	\begin{flushright}
		``Everybody should learn to program a computer,\\
		because it teaches you how to think.``\\
		Steve Jobs
	\end{flushright}
\end{epigrafe}
 % Epígrafe [OPCIONAL]
\begin{resumo}

Devido às diretrizes impostas pela Resolução Nº 7 de 2018 do Conselho Nacional de Educação (CNE) \cite{Resolucao-MEC:2018}, a curricularização da extensão se tornará obrigatória no ano de 2023. 
Tendo em vista que o processo de criação e manutenção de um programa ou projeto de extensão é demasiado demoroso e adicionado com a obrigatoriedade eminente, o objetivo deste trabalho é implementar uma ferramenta que ofereça suporte ao processo como um todo, permitindo desde a criação até a emissão de certificados para os participantes. 
Para isto ser possível, realizou-se primeiramente uma revisão sistemática na literatura cinza, em busca de ferramentas semelhantes, sendo extraído funcionalidades e aspectos mais pertinentes entre estas. 
Posteriormente, utilizando a lista alcançada, foi executado um levantamento (\textit{survey}) com os possíveis usuários finais da comunidade acadêmica da UNIPAMPA, com objetivo de classificar por ordem de importância as funcionalidades, além disso, permitindo com que os participantes fornecessem sugestões relacionadas as mesmas, ou até mesmo sugerindo novas. 
Os resultados foram analisados e iniciou-se a produção da solução proposta, uma ferramenta baseada na Web que auxiliará no esforço manual requerido nos processos relacionados a atividades de extensão. 
O desenvolvimento foi realizado por dois alunos de graduação, dividindo a carga em frontend e backend, este trabalho se concentra na área do backend.

\vspace{\onelineskip}
    
\noindent
\textbf{Palavras-chave}: Ferramenta. Survey. Literatura Cinza. Backend. Extensão. Atividade Extensionista. Comunidade. Universidade.

\end{resumo}
 % Resumo [OBRIGATÓRIO]
\begin{resumo}[Abstract]
 Due to the guidelines imposed by Resolution No. 7 of 2018 of the National Education Council (CNE) \cite{Resolucao-MEC:2018}, the curricularization of the extension will become mandatory in 2023. Given that the process of creation and maintenance of an extension program or project is too time-consuming and added to the imminent obligation, the objective of this work is to implement a tool that supports the process as a whole, allowing from the creation to the issuance of certificates for the participants. For this to be possible, a systematic review was first carried out in the gray literature, in search of similar tools, extracting the most relevant features and aspects among them. Subsequently, using the list reached, a survey was carried out with the possible end users of the academic community of UNIPAMPA, with the objective of classifying the functionalities in order of importance, in addition, allowing the participants to provide suggestions related to them, or even suggesting new ones. The results were analyzed and the production of the proposed solution began, a web tool that will assist in the manual effort required in the processes related to extension activities. The development was carried out by two undergraduate students, dividing the load into front-end and back-end, this work focuses on the back-end area.

 \vspace{\onelineskip}
 
 \noindent 
 \textbf{Key-words}: Tool. Survey. Grey Literature. Backend. Outreach. Community. University.
\end{resumo}
 % Abstract (resumo em inglês) [OBRIGATÓRIO]

% Figuras/Ilustrações [OPCIONAL]
\pdfbookmark[0]{\listfigurename}{lof}
\listoffigures*
\cleardoublepage

% Tabelas [OPCIONAL]
\pdfbookmark[0]{\listtablename}{lot}
\listoftables*
\cleardoublepage

% Siglas [OPCIONAL] (veja o pacote acro e os exemplo acima)
\pdfbookmark[0]{\listadesiglasname}{loa}
\printacronyms[include=acronimos,name=\listadesiglasname,heading=chapter*]
\cleardoublepage

% Sumário
\pdfbookmark[0]{Table of contents}{toc}
\tableofcontents*
\cleardoublepage

\textual

% Fundamentação Teórica (Background) - CAP de extensão universitaria, CAP de curricularização da extensão, soluções/ferramentas de apoio a extensão (spoiler) se basear nas leis federais, resoluç~oes da unipampa (se aprofundar na previa do antreprojeto) como que a extensão funciona no Brasil, como é implantada nas universidades, o que foi a lei de curricularização, capitulo para falar da unipampa cidadã (Geral sobre extensão, tipos, perfis de pessoas, sempre com funcamentação. Programas e projetos de extensão na Unipampa (para demonstrar como é importante dentro da faculdade, impacto da ferramenta, graficos, valores)

% Literatura Cinza - protocolo, resultados, lista de ferramentas encontradas -> lista preliminar de requisitos. 
    % Falar sobre as ferramentas finais, Screenshot, analise, resumo sobre elas, falar sobre cada uma delas

% Survey - Input = lista preliminar da literatuza, validação com o usuarios finais agregando novos resultados e ranqueandoos
    % protocolo, questionario, requisitos propostos, resultados, analise

% Extensionly - arquitetura da solução, devops (o que é devops, tecnologias), analise e projeto de software, artefatos de implementaç~ão, graficos, figuras, tecnologias de programação, modelo de dominio, diagrama de componentes, paradigma de programação, tudo relacionado a engenharia de software
    % mais detalhes tecnicos do que é a contribuição de cada um

%Considerações preliminares - ideia de publicar em eventos da area, visando publicações desses resultados, ERES agosto, mais eventos relacionados, congresso brasileiro de informatica na educação (SIMPOSIO SBIE, publicaç~ao de artigos relacionados a extensão, possiveis eventos alvo)

% Você pode dividir o seu texto em vários arquivos. Por exemplo, um para cada seção principal do
% trabalho: introducao.tex, relacionados.tex, metodologia.tex, experimentos.tex, conclusao.tex.

% Deadline 05 de agosto
% Introdução - Objetivo fazer um software ..., falar a limitação da unipampa em prover ferramentas (tem um SAP, que é so cadastro de projetos, mas o processo de gestão nao existe, so acompanhamento, nao gera certificado ...) por causa disso muita burocracia, em vez de fazer extensão opta por outra coisa menos burocratica
    % Deiar claro q é extenso, complexo, e que esta fazendo em duas maos, imagem de lego mostrando a coorelação
%==============================================================================
\chapter{INTRODUCTION}\label{introduction}
%==============================================================================
% Unipampa disponibiliza atividades extras, ensino, ...

% Atualmente a UNIPAMPA disponibiliza tres categorias de atividades extras, as de ensino, de pesquisa e de extensão. Atividades de ensino consistem no aprendizado do aluno de modo geral, podem ser cursos, aulas expositivas, atividades de monitoria, entre outras. As atividades de pesquisa são constituidas por tudo que  ́e relacionado a pesquisa em si, entre elas estão as iniciações cientificas, os Trabalhos de Conclusão de Curso (TCC), publicações de trabalhos em eventos, e assim por diante. Por  ́ultimo temos os projetos e programas de extens ̃ao, que s ̃ao o foco deste trabalho, e de acordo com o Plano de Desenvolvimento Institucional (PDI) de 2019, “A extensão assume o papel de promover a relação dialógica com a comunidade externa, pela democratização do acesso ao conhecimento acadêmico bem como pela realimentação das práticas universitárias a partir dessa dinâmica” [UNIPAMPA 2019].

\acl{UNIPAMPA} currently offers three categories of extra activities, teaching, research and outreach. 
Teaching activities consist of student learning in general, they can be courses, lectures, monitoring activities, among others. 
Research activities are constituted by everything that is related to research itself, among them are scientific initiations, \acp{TP}, publication of papers in events, and so on. 
Finally, we have the Outreach Projects and Programs, which are the focus of this work, and according to the 2019 \ac{IDP}, ``Outreach assumes the role of promoting a dialogic relationship with the external community, for the democratization of access to academic knowledge as well as for the feedback of university practices based on this dynamic'' \cite{PDI-Unipampa:2019-2023}.

%===========================================================

% O que é extensão

% Para explicar o que é extensão dentro de um ambiente acadêmico, será utilizado a Resolução Nº 104 de 2015 \cite{Resolucao-104:2015}, que esclarece a extensão como uma ação que incentiva a pesquisa e o desenvolvimento, aumentando o laço entre comunidade e universidade. As atividades extensionistas devem ter obrigatoriamente a participação da comunidade externa e promover o balanco entre atividades práticas e teóricas. Para realizar a classificação entre as atividades, são definidos quatro termos sendo eles: (1) Projetos, "conjunto de ações articuladas em torno de tema e objetivos comuns"; (2) Programas, "conjunto de projetos articulados, podendo contemplar mais de uma modalidade de ação (projeto, cursos, eventos)"; (3) Cursos, "atividades de formação"; (4) Eventos, "atividades de caráter artístico ou científico". Logo é preciso que alguns orgãos fiquem responsáveis pelo gerenciamento destas atividades, também definidos pela Resolução 104, são eles: (1) Pró-Reitoria de Extensão e Cultura (PROEXT); (2) Comissão Superior de Extensão (CSE); (3) Comissão Local de Extensão (CLE).

To explain what outreach is within an academic environment, Resolution No. 332 of 2021 will be used \cite{Resolucao-332:2021}, which clarifies \ac{OA} as an action that encourages research and development, increasing the bond between the community and \ac{HEI}. 
\acp{OA} must have the participation of the external community and promote a balance between practical and theoretical activities. 
To classify these outreach activities, four terms are defined, namely: 
(1) Projects, ``set of actions articulated around a common theme and objectives''; 
(2) Programs, ``set of articulated projects, which may include more than one type of action (project, courses, events)''; 
(3) Courses, ``training activities''; 
(4) Events, ``activities of an artistic or scientific nature''. 
Therefore, it is necessary for some bodies to be responsible for managing these activities, also defined by Resolution 104, they are: (1) \ac{PROEXT}; (2) \ac{CSE}; (3) \ac{CLE}.

%=============================================================

% O que é curricularização

% A curricularização da extensão descrita na Resolução Nº 7 de 2018 \cite{Resolucao-MEC:2018}, explicita que as atividades de extensão devem ter sua proposta, desenvolvimento e conclusão, devidamente registrados, documentados e analisados, de forma que seja possível organizar os planos de trabalho, as metodologias, os instrumentos e os conhecimentos gerados. 
% Também ordenando que as instituições de ensino deveriam incluir em seu Plano de Desenvolvimento Institucional (PDI), no mínimo 10% (dez por cento) da carga horária total do curso voltada para atividades de extensão, além de todos os termos relacionados, com prazo de até três anos, a contar da data de sua homologação. Tendo em vista esta demanda, a UNIPAMPA criou a Resolução \ac{CONSUNI} Nº 317 de 29 de abril de 2021, que implanta todas as diretrizes apresentadas pelo \ac{MEC}.

The curricularization of the outreach described in Resolution No. 7 of 2018 \cite{Resolucao-MEC:2018}, explains that \acp{OA} must have their proposal, development and conclusion, duly recorded, documented and analyzed, so that it is possible to organize work plans, methodologies, instruments and knowledge generated. 
Also ordering that educational institutions should include in their \ac{IDP}, at least 10\% (ten percent) of the total course load focused on \acp{OA}, in addition to all related terms, with a deadline of up to three years from the date of its approval. 
In view of this demand, \acl{UNIPAMPA} created \ac{CONSUNI} Resolution No. 317 of April 29, 2021, \cite{res317}, which implements all the guidelines presented by the \ac{MEC}.

%==============================================================
% Limitações da Unipampa

% Para controlar tudo isto é indispensável um software completo, e que seja de fácil uso, com que os usuários estejam confortáveis em usar e consigam completar suas tarefas utilizando-o, atualmente a UNIPAMPA apenas possuí um sistema chamado SAP, que serve apenas para o cadastro de projetos de extensão, mas não oferece suporte para a manutenção deles. 
% Por causa disto, acaba fazendo com que a burocracia se concentre fora do sistema, tornando este processo chato e demorado, os professores muitas vezes até desistem de realiza-lo, optando por outras atividades menos burocráticas.

To control all this, a complete software is indispensable, and that is easy to use, with which users are comfortable to use and can complete their tasks using it. 
Currently, \acl{UNIPAMPA} only has a system called \ac{SAP}, which serves for registration outreach projects, submit proposals to the public notices offered and manage the scholarship holders of the awarded notices, but does not contemplate the reach that we want to give. 
Because of this, it ends up making the bureaucracy concentrate outside the system, making this process boring and time-consuming, teachers often even give up doing it, opting for other less bureaucratic activities.

%==============================================================
% Unipampa cidadã

% Relacionado a este assunto, pouco tempo atrás foi divulgado a Instrução Normativa Nº18 \cite{unipampacidada}, que estipula as normativas do Programa Institucional ``UNIPAMPA Cidadã''. A qual é um programa de extensão que deverá ser composto por ações de cidadania e solidariedade, como campanha do agasalho, arrecadação de alimentos, suporte a asilos, etc., sendo obrigatório a sua oferta. Quando efetivada, em todos os cursos de graduação, deverá ser alocada uma carga horária mínima de 60 e máxima de 120.

Related to this matter, Normative Instruction No. 18 \cite{unipampacidada} was released a short time ago, which stipulates the norms of the Institutional Program ``UNIPAMPA Cidadã''. 
Which is an outreach program that should be composed of citizenship and solidarity actions, such as clothing campaign, food collection, support for asylums, etc., being mandatory to offer them. 
When effective, in all undergraduate courses, a minimum workload of 60 and a maximum of 120 must be allocated.

%------------------------------------------------------------------------------
\section{Motivation}\label{sec:motivation}
%------------------------------------------------------------------------------
% Trabalho manual -
% Necessidades da comunidade é dificil atender (falta de ferramenta) -
% Opção de professores por fazer algo menos burocratico
% Divulgação espalhada, muitos emails, alunos não veem
% Literatura cinza e nao encontramos a melhor que resolvesse por completo os problemas não apenas parcialmente, citar duas (pode ser as do anteprojeto e mais algumas da literatura) 
%==============================================================
% O processo de curricularização proposto pela Resolução Nº 317 \cite{res317}, se tornará obrigatório em 2023, tendo em vista o esforço que será necessário para completar demandas manualmente como cadastro, controle, emissão de certificados e ingresso dos participantes, implícitos em uma \ac{OCA}, fora proposto a criação de uma ferramenta de apoio na gestão destes projetos e programas de extensão, conseguindo assim diminuir a burocracia e agilizar o processo.

The process of curricularization of outreach proposed by Resolution Nº 317 \cite{res317}, will become mandatory in 2023, given the effort that will be required to manually complete demands such as registration, control, issuance of certificates and entry of participants, implicit in an \ac{OCA}, it was proposed to create a support tool in the management of these projects and outreach programs, thus managing to reduce bureaucracy and speed up the process.

%==============================================================

% A comunidade periodicamente entra em contato com a universidade para solicitar algum tipo de ação solidária, com esta demanda são geradas atividades extensionistas, podendo ser exercidas por alunos gerenciados por um professor responsável ou até mesmo dentro de uma disciplina de seus cursos. Mas esta comunicação não é a das mais intuitivas, não possuindo um sistema para gerencialas, leva a unica opção de ter que fazela por meio de ligações ou até mesmo presencialmente, isto é muito desanimador para a comunidade. Tendo em vista isso, uma das motivações do desenvolvimento da ferramenta é fortalecer este laço entre comunicade acadêmica e comunicade externa, permitindo que novas demandas sejam criadas na própria ferramenta.

The community periodically contacts the university to request some type of solidarity action, with this demand, \acp{OA} are generated, which can be carried out by students managed by a coordinator or even within a subject of their courses. 
But this communication is not the most intuitive, not having a system to manage them, it leads to the only option of having to do it through calls or even in person, this is very discouraging for the community. 
In view of this, one of the motivations for the development of the tool is to strengthen this link between academic communication and external communication, allowing new demands to be created in the tool itself.

%==============================================================
% Em relação a divulgação das atividades de extensão, hoje em dia são enviados emails para os alunos informando sobre novas oportunidades, mas geralmente as caixas de entrada dos alunos ja são bombardeadas por emails no dia a dia, levando ao desinteresse de ler todos eles. Por este motivo, com uma ferramenta que concentrasse todas as informações, oportunidades e novidades relacionadas a extensão, os alunos não precisariam mais se aventurar a pesquisar no seu mar de emails quando precisarem procurar por uma nova atividade, apenas recorreriam a ferramenta onde tudo ja esta organizado e pronto para ser utilizado.

Regarding the dissemination of \acp{OA}, nowadays emails are sent to students informing them about new opportunities, but usually students inboxes receives a lot of emails on a daily basis, leading to the lack of interest in reading all of them. 
For this reason, with a tool that concentrated all the information, opportunities and news related to the outreach.
Hence, the students would no longer need to venture into their sea of emails when they need to look for a new activity, they would just resort to the tool where everything is already organized and ready to use.

%==============================================================
% Outro motivador que incentivou o desenvolvimento desta ferramenta é que a partir da revisão na literatura cinza que foi conduzida, não foram encontradas ferramentas que solucionavam por completo os problemas relacionados a estes processos. Algumas ferramentas apresentavam funcionalidades e detalhes que outras não apresentavam e vice-versa, mas que juntas construiriam uma ferramenta íntegra.

Another motivator that encouraged the development of this tool is that from the review in the grey literature that was conducted, no tools were found that completely solved the problems related to these processes. Some tools had features and details that others did not and vice versa, but together they would build a complete tool.

%------------------------------------------------------------------------------
\section{Objectives}\label{sec:objectives}
%------------------------------------------------------------------------------
% Fazer ferramenta que englobe todos os passos
% Deixar mais simples o processo
% Centralização de informações
% Redução do trabalho manual
% Fazer com que a comunidade utilize a ferramenta para sugerir atividades
% Facilitar a comunicação entre aluno e professor
% 

% Tendo em vista o que foi apresentado, o objetivo geral do tema deste TCC, é o desenvolvimento do \textit{backend} de uma ferramenta que servira de apoio na gestão de programas e projetos de extensão, reproduzindo e auxiliando em todos os processos referentes a este assunto, desde sua criação até a geração de certificados quando for finalizada. O intuito é diminuir o esforço e tempo gasto pelos envolvidos, nestas etapas manuais do processos, além disso, permitir que seja construido um novo canal de comunicação entre comunidade acadêmica e comunidade externa, permitindo sugestões de demandas para atividades de extensão diretamente na ferramenta.

In view of what has been presented, the research aim of the theme of this term paper is the development of the backend of a tool that will serve as support in the management of outreach programs and projects, reproducing and assisting in all processes related to this demand, from its creation to the generation of certificates when it is finalized. 
The aim is to reduce the effort and time spent by those involved in these manual steps of the process. 
In addition to allowing a new communication channel to be built between the academic community and the external community, allowing suggestions for demands for \acp{OA} directly in the tool.

The following research objectives were established in order to achieve the research goal:
\begin{itemize}
  \item Systematically review grey literature works and products to find comparable solutions, collecting the first batch of requirements.
  \item Elaborate a survey, according to \textcite{kasunic2005designing}, in order to identify new system requirements and to better comprehend the requirements of the intended users.
  \item Create concrete tasks and an implementation roadmap by analyzing the findings and refining the elicited requirements.
  \item Study the most relevant tools, programming languages, and frameworks to create a stack that offers excellent code maintainability, architecture, and performance.
  \item Create an operational \acl{MVP} (\ac{MVP}) of the system that initially implements the most important requirements that have been gathered and refined.
\end{itemize}

% Portanto, na Tabela \ref{tbl:tableObjectives} é apresentada a síntese dos objetivos gerais e específicos, bem como o assunto, o tema, o problema de pesquisa e a hipótese de solução.

Therefore, the Table \ref{tbl:tableObjectives} presents the synthesis of the research aims and objectives, as well as the subject, the study, the research question (problem) and the solution hypothesis.

\input{componentes/1-objectives}

%------------------------------------------------------------------------------
\section{Contribution}\label{sec:contribution}
%------------------------------------------------------------------------------

% Trabalho complexo
% Dois alunos de graduação
% Back e Front, este é back
% Imagem

% A proposta para esta ferramenta é projetada para a participação de dois alunos, Igor Dalepiane da Costa e Lucas Alexandre Fell, pois a sua complexidade é alta, justificando este desenvolvimento em dupla.

% A principal contribuição deste estudo é o desenvolvimento MVp para o backend de uma ferramenta to support and automate the whole process of \aclp{OCA} in the university. 
% Tambem é esperado a geração de dois artefatos, uma revisão sistematica na literatura cinza, para levantar ferramentas semelhantes a proposta, e coletar as suas funcionalidades e detalhes mais importantes. 
% Também será executado um survey com possiveis usuarios finais da ferramenta, para a classificação por relevancia dos requisitos levantados e para entender mais sobre a real necessidade deste publico.

The main contribution of this study is the \ac{MVP} development for the backend of a tool to support and automate the whole process of \aclp{OCA} in the university.
It is also expected the generation of two artifacts, a systematic review in the gray literature, to raise tools similar to the proposal, and collect their most important features and details.
A survey will also be carried out with possible end users of the tool, to classify the requirements raised by relevance and to understand more about the real need of this public.

The proposal for this tool is designed for the participation of two students because its complexity is high, justifying this double development, Igor Dalepiane da Costa, being responsible for the backend and Lucas Alexandre Fell, responsible for the frontend. Regarding the artifacts created to support the research, such as the survey and the systematic review of the grey literature, they were all completed in collaboration by both authors and are not specifically related to any one piece of work.

% Para isto foi feito a divisão entre o desenvolvimento do  \textit{backend} e  \textit{front-end}, o primeiro sendo desenvolvido por pelo proponente deste trabalho e o segundo pelo outro estudante. Para melhor visualização foi desenvolvido um \textit{feature model} com a divisão exata das tarefas que serão desempenhadas por cada um dos estudantes, representado na Figura 1. São contribuições deste trabalho:



% For this, the division was made between the development of \textit{backend} and \textit{front-end}, the first being developed by the proponent of this work and the second by the other student. For better visualization, a feature model was developed with the exact division of the tasks that will be performed by each of the students, represented in Figure 1. Contributions of this work:

% \begin{itemize}
%     \item Pesquisa entre os docentes da universidade sobre a real necessidade deste instrumento de organização dos processos das \acp{OCA}.
%     \item Desenvolvimento do  \textit{backend} do sistema, englobando todos os processos que serão disponibilizados pela ferramenta, explicitados na Figura 1.
% \end{itemize}

% \begin{itemize}
%     \item Research among university professors on the real need for this instrument to organize the \acp{OCA} processes;
%     \item Development of the \textit{backend} of the system, encompassing all the processes that will be made available by the tool, explained in \Cref{fig:featuremodel}.
% \end{itemize}

% \usetikzlibrary{angles,shadows.blur,positioning,backgrounds}
\forestset{
%   declare count register=disjuncts from,
%   disjuncts from'=0,
%   declare count register=colega from,
  colega colour/.code={\colorlet{colegacol}{#1}},
  autor colour/.code={\colorlet{autorcol}{#1}},
  draw colour/.code={\colorlet{drawcol}{#1}},
%   colega colour=gray,
%   autor colour=white,
  draw colour=black,
  /tikz/colega/.style={fill=white, draw=drawcol},
  /tikz/autor/.style={fill=green!25, draw=drawcol},
  disjunct/.style={
    tikz+={\path (.parent) coordinate (A) -- (!u.children) coordinate (B) -- (!ul.parent) coordinate (C) pic [fill=drawcol] {angle};}
  },
  disjunction tree/.style={
    where={isodd(n_children())}{
      for n={int((n_children()+1)/2)}{calign with current},
    }{
      calign=midpoint,
    },
    before typesetting nodes={
      for nodewalk={
        filter/.wrap pgfmath arg={{level>=##1}{n_children()>1}}{(disjuncts_from)}
      }{
        or,
      },
      tikz+={
        [font=\sffamily]
        \node (l) [anchor=east] at (current bounding box.north west) {Legenda};
        \foreach \i/\j [remember=\i as \k (initially l)] in {autor/Autor,colega/Colega}
        {
          \node (\i) [below=20pt of \k.north, anchor=north, text centered, \i, minimum width=5pt,] {};
          \node (\j) [right=5pt of \i.center, anchor=west] {\j};
        };
        \draw [drawcol] (or.south west) coordinate (A) -- (or.north) coordinate (B) -- (or.south east) coordinate (C) pic [fill=drawcol, angle radius=5pt] {angle};
        \node (c) [below=0pt of Colega.south] {};
        \scoped[on background layer]{\node [draw, fill=white, blur,fit=(l) (Autor) (Colega) (c)] {};}
      },
    },
    for tree={
      parent anchor=children,
      child anchor=parent,
      l'+=10mm,
      fill=gray!15,
      text height=2ex,
      text depth=.5ex,
      font=\sffamily,
    }
  },
}

\begin{figure}[!htb]
\label{fig:featuremodel}
\begin{forest} 
    disjunction tree,
    draw colour=darkgray,
    [Extensionly, fill=blue!15, draw=drawcol
        [Back-end, autor,
            [Manter ACEs, autor]
            [Testes, autor
                [Funcionais, autor]
                [Unitários, autor]
            ]
            [Certificados, autor, name=c
                [Geração, autor]
                [Validação, autor]
                [Solicitação, colega]
            ]
        ]
    ]
\end{forest}
\end{figure}

%------------------------------------------------------------------------------
\section{Organization}\label{sec:organization}
%------------------------------------------------------------------------------

This document is organized according to the following:

\begin{itemize}
    \item  \textbf{Chapter 2: Methodology:} Details of the the methodology adopted during the search, along with it's classification and research schedule.
    \item  \textbf{Chapter 3: Background:} Details of main concepts related to this work, such as, resolutions and \acp{OA}.
    \item  \textbf{Chapter 4: Grey Literature:} This chapter presents in more detail the review performed in the grey literature to find similar tools.
    \item  \textbf{Chapter 5: Survey:} Provides details about the survey performed, its protocol and results.
    \item  \textbf{Chapter 6: Extensionly:} Provides details of the design and implementation
of the proposed tool.
    \item  \textbf{Chapter 7: Conclusions:} This chapter presents the partial conclusions about this study.
\end{itemize}



 % 15/07
% Metodologia 3-4 pags, desenho da pesquisa (BPMN, fases e atividades, exemplos nos tcs)
% Cronograma - datas, quando vai ser feito, desde o inicio do projeto la no anteprojeto, proposta, o que foi feito em MPA, o que foi tento entre um semestre e outro, o que ficou de resultado para o TC1
%==============================================================================
\chapter{METHODOLOGY}\label{methodology}
%==============================================================================

% Neste capitulo sera apresentado a metodologia, técnicas e procedimentos que foram utilizadados no decorrer deste estudo. Comecando com a \Cref{sec:met-intro}, onde é apresentada o significado de pesquisa. Na \Cref{sec:met-classification}, será feita a classificação da pesquisa utilizando termos e definições apresentados por \cite{Prodanov:2013}. Em seguida na \Cref{sec:met-design}, é apresentado como este estudo foi conduzido, junto com o desenho de pesquisa, o cronograma da pesquisa, com datas limites e espaços de tempo se encontra na \Cref{sec:met-schedule}.

In this chapter, we presente the methodology, techniques and procedures that were used in the course of this study. 
Starting with \Cref{sec:met-intro}, where we presente the context of the research. 
In \Cref{sec:met-classification}, the search will be classified using terms and definitions presented by \citeonline{Prodanov:2013}. 
Then in \Cref {sec:met-design}, it is presented how this study was conducted, along with the research design, the research schedule, with deadlines and time spaces is in \Cref{sec:met-schedule}.

\section{Introduction}\label{sec:met-intro}

% Para que em um estudo os objetivos sejam alcançados com sucesso, a pesquisa cientifica é considerada muito importante pelas suas contribuições. De acordo com \cite{pingping_yulan_2013}, o objetivo de uma pesquisa é explorar a presente situação e o desenvolvimento do mundo, sob os objetivos estabelecidos anteriormente e planos de conhecimento desconhecidos.

In order for the objectives of a study to be successfully achieved, scientific research is considered very important for its contributions. 
According to \cite{pingping_yulan_2013}, the purpose of a research is to explore the present situation and development of the world, under the previously set goals and unknown knowledge plans.

% Existem diversas maneiras com que a pesquisa cientifica pode ser conduzida e é designado aos pesquisadores definirem qual utilizar, visando a maior relevância nos seus resultados. É de muita importância que sejam escolhidos como base para o desenvolvimento da pesquisa, autores e estudos de sucesso, pois como \cite{dampier_wilson} diz, "O método científico é um processo através do qual cada avanço para o estado da arte é construído sobre verdades conhecidas e avanços anteriores".

There are several ways in which scientific research can be conducted and it is assigned to researchers to define which one to use, aiming at the greatest relevance in their results. 
It is very important that they are chosen as the basis for the development of research, authors and successful studies, because as \citeonline{dampier_wilson} says, the advances made previously and the known truths, serve as a basis for the advances of the scientific method.


\section{Research Classification}\label{sec:met-classification}

% A classificação desta pesquisa se deu de acordo com as definições feitas por \cite{Prodanov:2013}, na \Cref{fig:research-classification} a classificação da pesquisa esta separada por quatro grupos, cada um com suas respectivas categorias, são os grupos: 
% \begin{inparaenum}[(1)]
%   \item According to the Approach;
%   \item According to the Nature;
%   \item According to the Objectives;
%   \item According to the Procedures.
% \end{inparaenum}
% Na imagem os retangulos preenchidos com cor azul representam os que se aplicam a presente pesquisa. 

The classification of this research was given according to the definitions made by \citeonline{Prodanov:2013}, in \Cref{fig:research-classification} the classification of the research is separated by four groups, each with its respective categories, are the groups: 
\begin{inparaenum}[(i)]
  \item According to the \textbf{Approach};
  \item According to the \textbf{Nature};
  \item According to the \textbf{Objectives};
  \item According to the \textbf{Procedures}.
\end{inparaenum}
In \Cref{fig:research-classification} the rectangles filled with blue color represent those that apply to this research. 

% Iniciando com o ponto de vista da natureza, esta se encaixa em \textbf{Applied Research}, pois busca aplicar novos conhecimentos gerados em problemas objetivos, envolvendo verdades, interesses e demandas locais. Trazendo para a realidade deste trabalho, os conhecimentos gerados se refere a todos os dados levantados relacionados a extensão no decorrer do estudo, e o problema objetivo é a burocracia envolvida nos projetos de extensão.

Starting with the point of view of nature, this fits into \textbf{Applied Research}, as it seeks to apply new knowledge generated in objective problems, involving truths, interests and local demands. 
Bringing to the reality of this work, the knowledge generated refers to all the data collected related to outreach in the course of the study, and the objective problem is the bureaucracy involved in \acp{OA}.

% Tendo em vista os objetivos, esta é classificada como \textbf{Exploratory Research} pois para alcancar os objetivos definidos, pesquisas na literatura cinza e questionários com pessoas relacionadas ao assunto, foram executados. Sendo assim, utilizando o que ja existe como base, busca-se construir uma nova solução aprimorada.

In view of the objectives, this is classified as \textbf{Exploratory Research} because to achieve the defined objectives, research in the grey literature and questionnaires with people related to the subject were performed. 
Thus, using what already exists as a basis, we seek to build a new improved solution.

% Em relação aos procedimentos técnicos, se aplica \textbf{Case Study}, pois busca coletar informações de individuos, ferramentas, processos, relacionados com o tema principal utilizando métodos \textbf{Qualitatives}, para ser possível colocar os resultados e gráficos e analisa-los, and \textbf{Quantitavive}, permitindo o entendimento mais profundo do que foi respondido. A classificação \textbf{Survey} também se aplica, por ser esta uma das maneiras de coleta de informação utilizada pelos pesquisadores, antes da execução com o público respondente, um teste piloto foi conduzido, para validar organização, completude, coerência e outros pontos do questionário, mais deste será discutido na \Cref{sec:5}.

In relation to technical procedures, \textbf{Case Study} is applied, as it seeks to collect information from individuals, tools, processes, related to the main theme using \textbf{Qualitative} methods, to be able to place the results and graphs and analyze them, and \textbf{Quantitavive}, allowing a deeper understanding of what was answered. 
The \textbf{Survey} classification also applies, as this is one of the ways of collecting information used by researchers. 
Before the survey execution with the participants, a pilot test was conducted to validate organization, completeness, coherence and other points of the questionnaire, more of this will be discussed in \Cref{sec:5}.

% Por último, o trabalho também se encontra classificado como \textbf{Documentrary Research}, por usar como base de conhecimento, materiais que ainda não receberam um tratamento analítico, como resultados de pesquisas na internet, o assunto sobre literatura cinza sera explicado melhor na \Cref{sec:4}.

Finally, the study is also classified as \textbf{Documentary Research}, for using as a knowledge base, materials that have not yet received an analytical treatment, such as internet search results, the subject of grey literature will be better explained in \Cref{sec:grey_literature}.

\begin{figure}[!htb]
  \caption{Research Classification}
  \label{fig:research-classification}
  \begin{center}
    \includegraphics[width=14cm]{img/pesquisaSurvey.png}
  \end{center}
  \fonte{Adapted from \cite{Prodanov:2013}.}
\end{figure}

\section{Research Design}\label{sec:met-design}

% Na \Cref{fig:research-design} está representada o fluxograma seguido no decorrer desta pesquisa, as atividades nele posicionadas estão divididas entre cinco fases:
In \Cref{fig:research-design} is represented the flowchart followed in the course of this research, the activities placed in it are divided into five phases:
\begin{inparaenum}[(1)]
  \item Information gathering;
  \item Partial development;
  \item Development;
  \item Evaluation;
  \item Publish.
\end{inparaenum}

% A primeira fase, \textbf{Information gathering}, é focada em organizar estruturas de pesquisa, questionarios, priorização de informações, aprendizado sobre o tema da pesquisa. Principalmente direcionada a produzir dois importantes artefatos da pesquisa, a revisão na literatura cinza e o survey com possíveis usuários finais.

The first phase, \textbf{Information Gathering}, is focused on organizing research structures, questionnaires, prioritization of information, and learning about the research topic. Mainly aimed at producing two important artifacts of the research, the review in the grey literature and the survey with possible end users.

% Seguindo para a segunda fase, \textbf{Partial Development}, onde foi decidido entre os envolvidos no projeto, que não seria viável implementar todo a ferramenta neste primeiro momento, então apenas algumas funcionalidades mais importantes e que ja seriam suficientes para um produto em estado inicial, seriam desenvolvidas.
% Dentro da fase de \textbf{Publish}, os dois Term Papers serão escritos e defendidos, ocorrendo de forma paralela ao desenvolvimento da ferramenta, majoritariamente acontecendo na fase de \textbf{Development}.

Moving on to the second phase, \textbf{Partial Development}, where it was decided among those involved in the project, that it would not be feasible to implement the entire tool at this first moment, so only some more important functionalities and that would already be sufficient for a \ac{MVP} \cite{Lenarduzzi:2016}, would be developed.
Within the \textbf{Publish} phase, the two \acp{TP} will be written and defended, occurring in parallel to the development of the tool, mostly happening in the \textbf{Development} phase.

% Após ja existir uma versão estável da ferramenta, onde usuários possam utiliza-la, esta será disponível para uso real, permitindo que um curso de extensão da Unipampa seja cadastrado e abrindo vagas para inscrições de participantes, com isto na fase de \textbf{Evaluation}, serão coletados os feedbacks, analisado os resultados e melhorias na ferramenta serão feitas.

After there is a stable version of the tool, where users can use it, it will be available for real use, allowing UNIPAMPA's outreach activities to be registered and opening vacancies for participant or volunteer registrations, with this in the phase of \textbf{Evaluation}, feedbacks will be collected, analyzed the results and improvements in the tool will be made.

\begin{figure}[htb]
  \caption{Research Design}\label{fig:research-design}
  \begin{center}
    \includegraphics[width=16cm]{img/researchDiagram.png}
  \end{center}
  \fonte{Author.}
\end{figure}

\section{Research Schedule}\label{sec:met-schedule}

% Para facilitar a visualização de como as atividades se deram ao decorrer do tempo, na \Cref{tbl:schedule} é apresentado todo o cronograma do que foi planejado desde o levantamento de informações até a defesa do Term Paper II.

To facilitate the visualization of how the activities took place over time. 
\Cref{tbl:schedule} presents the entire schedule of what was planned from the collection of information to the defense of \acl{TP} II.

\input{componentes/2-schedule}


\section{Chapter Summary}\label{sec:met-summary}

% Neste capitulo foi apresentado o significado de metodologia, e como ela pode ser classificada dentro de um ambito cientifico, juntamente com quais termos que se aplicam ao presente Term Paper. Além disso foi apresentado o design de pesquisa contendo os passos realizados pelo autor, como também os que serão dados.

In this chapter we have presented the meaning of methodology, and how it can be classified within a scientific scope, along with what terms apply to this \ac{TP}. 
In addition, the research design was presented containing the steps taken by the author, as well as those that will be given.

%------------------------------------------------------------------------------

%------------------------------------------------------------------------------
% Migrar o anteprojeto
%  Falar sobre a literatura cinza para identificar as ferramentas, atualizar a figura
%  Atualizara imagem colocar o survey e na classificaç~ao de pesquisa adicionar o survey. Retirar pesquisa documental
%  Olhar os outros TCCs para exemplo
%  Detalhar mais o que é cada caixa do desenho de pesquisa
%  Detalhar mais cada caixinha da figura 3
% Imagens: Cronograma desde 2021/2, atualizar (segundo semestre vai ate julho) (prox ano comeca em julho e vai ate janeiro) 
  % 20/07
% Fundamentação Teórica (Background) - CAP de extensão universitaria, CAP de curricularização da extensão, soluções/ferramentas de apoio a extensão (spoiler) se basear nas leis federais, resoluç~oes da unipampa (se aprofundar na previa do antreprojeto) como que a extensão funciona no Brasil, como é implantada nas universidades, o que foi a lei de curricularização, capitulo para falar da unipampa cidadã (Geral sobre extensão, tipos, perfis de pessoas, sempre com funcamentação. Programas e projetos de extensão na Unipampa (para demonstrar como é importante dentro da faculdade, impacto da ferramenta, graficos, valores)

%==============================================================================
\chapter{BACKGROUND}\label{background}
%==============================================================================
% 1 ou 2 paragrafos 

% ============================================================================
% Neste capitulo são discutidos assuntos assuntos que complementam o objetivo do presente trabalho, ajudando no entendimento das politicas e resoluções envolvidas. 
% Na \Cref{sec:3.1} sera apresentada a politica nacional de extensão, que vale para todo o Brasil sobre os objetivos que a extensão universitária tem em relação a comunidade academica e externa. 
% Logo em seguida na \Cref{sec:3.2} a visão de como a Unipampa se adaptou para receber estas novas regras. 
% Após, na \Cref{sec:3.2.1} a diferenca entre programas e projetos de extensão sera apresentada, seguido por uma explicação mais detalhada sobre o projeto 'Unipampa Cidadã' na \Cref{sec:3.2.2}. 
% A \Cref{sec:3.3} releva algumas ferramentas relacionadas ao assunto do trabalho, seus pontos em comum e uma descrição em alto nível. 
% Por fim na \Cref{sec:3.4} um resumo geral sobre o capitulo é apresentado.

This chapter discusses subjects that complement the objective of this work, helping to understand the policies and resolutions involved.
In \Cref {sec:3.1} the national outreach activity policy will be presented, which is valid for all of Brazil on the objectives that university outreach has in relation to the academic and external community.
Then in \Cref{sec:3.2} the vision of how Unipampa has adapted to receive these new rules.
After that, in \Cref {sec:3.2.1} the difference between outreach programs and projects will be presented, followed by a more detailed explanation about the ``Unipampa Cidadã'' project in \Cref{sec:3.2.2}.
The \Cref{sec:3.3} highlights some tools related to the subject of the work, their commonalities and a high-level description.
Finally in \Cref{sec:3.4} a general summary of the chapter is presented.
% ============================================================================

\section{National Outreach Policy}\label{sec:3.1}
% Descrever a politica em si
% ============================================================================
% Sabemos que a extensão universitária é uma area de grande importancia para a comunidade academica e externa, também sendo uma ferramenta de conexão entre professores, alunos e população, tendo muito impacto na formação de um estudante. 
% Para fortalecer os objetivos que a extensão universitaria tem dentro deste universo, o Fórum de Pró-Reitores de Extensão das Universidades Públicas Brasileiras (FORPROEX), acrescentou a versão antiga do documento da Politica Nacional de Extensão, publicado em 1999, com situações atuais e desafios encontrados nos ultimos anos. 
% A nova versão do documento, \cite{politicaNacional}, dentro dos seus objetivos, temos como exemplo os que seguem:

It is well-known that university outreach is an area of great importance for the academic and external community, also being a tool for connecting professors, students and the community, having a high impact on the soft skills a student formation. 
To strengthen the objectives that university outreach has within this universe, the \acl{FORPROEX} (\ac{FORPROEX}), updated the old version of the National Outreach Policy document, published in 1999, with current situations and challenges found in recent years.
The new version of the document, \cite{politicaNacional}, within its objectives, has as an example the following:
% ============================================================================

% ============================================================================
% \begin{itemize}
%     \item Conquistar o reconhecimento da extensão universitária, como uma ferramenta essencial para a universidade pública.
%     \item Garantir que a extensão seja a solução para qualquer tipo de problema social enfrentado pelo país.
%     \item Defender o financiamento de programas e projetos de extensão para que estes mantenham o seu funcionamento.
%     \item Promover a conscientização ambiental e sustentavel em projetos extensionistas no Brazil.
%     \item Além de nacionalmente, promover a solidariedade internacionalmente, abrangindo a aréa de impacto das ações extensionistas.
% \end{itemize}

\begin{itemize}
    \item Achieve the recognition of university outreach activities as an essential tool for the public university;
    \item Ensure that the outreach activity is the solution to any type of social problem faced by the country;
    \item Defend the funding of outreach programs and projects so that they can continue to function;
    \item Promote environmental and sustainable awareness in outreach projects in Brazil;
    \item Promote solidarity both nationally and internationally, covering the area of impact of outreach actions.
\end{itemize}
% ============================================================================

% ============================================================================
% Servindo como base para as universidades, o documento ``Referenciais para a construção de uma Política Nacional de Extensão nas ICES'' \cite{referenciaisPolitica}, 
% discute um pouco sobre a duvida da classificação de uma atividade academica como extensionista ou não, mas deixando como fato a seguinte frase ``Se a dimensão teórica da Extensão tende à maior rigidez - no sentido que precisa guardar princípios, retomar referenciais, 
% dialogar com outros documentos institucionais – a dimensão prática possibilita maior flexibilidade, originando uma considerável diversidade de ações''. 
% Este documento também destaca a importancia da integração da extensão com a pesquisa e ensino, com discussões de cunho social e efeitos dos resultados na sociedade.

Serving as a basis for universities, the document ``Referential for the construction of a National Outreach Policy in \aclp{ICES} (\ac{ICES})'' \cite{referenciaisPolitica},
discusses a little about the doubt of classifying an academic activity as outreach or not, but leaving as a fact the following sentence ``If the theoretical dimension of university outreach tends towards greater rigidity - in the sense that it needs to keep principles, resume references,
dialogue with other institutional documents – the practical dimension allows for greater flexibility, giving rise to a considerable diversity of actions'' \cite[p.43]{referenciaisPolitica}.
This document also highlights the importance of integrating outreach with research and teaching, with discussions of a social nature and the effects of the results on society.
% ============================================================================

% ============================================================================
% No documento supracitado, conjuntamente são aprofundados nove tipos de ações de extensão possíveis, cada um com suas peculiaridades, dividindo-as  em ações diretas de extensão e ações que permitem a integração entre extensão e ensino e extensão e pesquisa. 

As policy aforementioned, nine possible outreach activity types are discussed in depth, each with its peculiarities, dividing them into direct outreach actions and actions that allow the integration between outreach and teaching or outreach and research.

% ============================================================================

%https://sites.unipampa.edu.br/proext/documentos/politica-nacional-de-extensao/
% Plano Nacional de Educação 2014-2024 (Lei 13.005/2014)
% https://curricularizacaodaextensao.ifsc.edu.br/files/2016/06/1_Artigo_Curricularizaca_da_Extensao_Universitaria_no_Brasil.pdf
% FOREXT. Extensão nas Instituições Comunitárias de Ensino Superior: referenciais para a construção de
% uma Política Nacional de Extensão nas ICES (2013). Recuperado em 12 de março, 2015, em
% <http://www1.pucminas.br/imagedb/documento/DOC_DSC_NOME_ARQUI20150309182334.pdf>
% FORPROEX. Política Nacional de Extensão Universitária (2012). Recuperado em 12 de outubro, 2014,
% de <http://www.renex.org.br/documentos/2012-07-13-Politica-Nacional-de-Extensao.pdf>

\subsection{\acl{OA} Curricularization in Higher Education}\label{sec:3.1.1}
% Descrever aqui a ideia geral de como implementar isso dentro dos cursos pelas diferentes IESs
% ============================================================================
% Entrando no âmbito do ensino superior foi criado a Resolução Nº 7, de 18 de Dezembro de 2018 \cite{ministerioSuperiorExtensao}, aonde ela instituí diretrizes, princípios, fundamentos e procedimentos para a extensão na educação superior brasileira.
% Desta maneira foi regulamentado que as atividades de extensão serão disponibilizadas na forma de componentes curriculares para os cursos.

Entering the scope of higher education, Resolution No. 7, of December 18, 2018 \cite{ministerioSuperiorExtensao} was created, where it established guidelines, principles, foundations and procedures for university outreach in Brazilian higher education.
In this way, it was regulated that the \acp{OA} will be made available in the form of curricular subjects for the courses.
% ============================================================================

% ============================================================================
% Neste documento também é determinado que as atividades de extensão devem compor no minimo 10\% (dez por cento) de toda a carga horária dos cursos de graduação, sendo elas caracterizadas como uma atividade intervencionista que envolva diretamente a comunidade externa e esteja relacionada com a formação estudantil.

In this document, it is also determined that \acp{OA} must make up at least 10\% (ten percent) of the entire workload of undergraduate courses, being characterized as an interventionist activity that directly involves the external community and is related to student training.
% ============================================================================

% ============================================================================
% Outro ponto importante levantado, tem relação com a autoavaliação das atividades de extensão, para ocorrer ao aperfeicoamento constante da mesma. Nesta avaliação deverá ser incluido a identificação da pertinência da utilização das atividades de extensão na
% creditação curricular, a contribuição para o cumprimento dos objetivos do Plano de Desenvolvimento Institucional e dos Projetos Pedagógico dos Cursos e por fim a apresentação dos resultados conquistados em relação ao público participante.

Another important point raised is related to the self-assessment of outreach activities, in order to constantly improve it. 
This evaluation should include the identification of the relevance of the use of \acp{OA} in curricular accreditation, the contribution to the fulfillment of the objectives of the \ac{IDP} and the Pedagogical Projects of the Courses and, finally, the presentation of the results achieved in relation to the participating public.
% ============================================================================

% ============================================================================
% Todas as atividades de extensão também deverão ser registradas conforme as regras citadas na mesma resolução \cite{ministerioSuperiorExtensao}, devendo conter o planejamento de suas atividades internas, as estratégias para a autoavaliação, proposta, desenvolvimento e conclusão, estes devem estar devidamente registrados e analisados para poder ser feito a organização de seus planos de trabalho. 

All \acp{OA} must also be registered according to the rules mentioned in the same resolution \cite{ministerioSuperiorExtensao}, and must contain the planning of their internal activities, strategies for self-assessment, proposal, development and conclusion, these must be duly registered and analyzed in order to be able to organize your work plans.
% ============================================================================

% ============================================================================
% Por fim, a resolução supramencionada determina que ``As instituições de ensino superior terão o prazo de até 3 (três) anos, a contar da data de sua homologação, para a implantação do disposto nestas Diretrizes.''

Finally, the aforementioned resolution determines that ``Higher education institutions will have a period of up to 3 (three) years, counting from the date of their approval, to implement the provisions of these Guidelines.''
% ============================================================================

\section{\acl{OA} Curricularization in \acl{UNIPAMPA}} \label{sec:3.2}
% Descrever a visão da unipampa 
% https://sites.unipampa.edu.br/proext/documentos/normas-de-extensao-da-unipampa/
% https://sites.unipampa.edu.br/proext/files/2021/05/res-317_2021-politica-de-extensao.pdf
% - RESOLUÇÃO CONSUNI/UNIPAMPA Nº 317, DE 29 DE ABRIL DE 2021
% https://sites.unipampa.edu.br/proext/files/2021/12/sei_unipampa-0700488-resolucao-consuni.pdf
% - RESOLUÇÃO CONSUNI/UNIPAMPA Nº 332, DE 21 DE DEZEMBRO DE 2021

% Citar os cases de sucesso como ES e outros cursos de Uruguaiana

% ============================================================================
% Estando na visão da Unipampa, ela como todas as universidades federais, deve ter uma resolução voltada para a normatização para as atividades de extensão de modo geral apresentando o que elas são, seu publico alvo, objetivos etc. 
% Tendo em vista isso a Unipampa na Resolução CONSUNI/UNIPAMPA Nº 332 de 2021, \cite{Resolucao-332:2021}, determina os tipos de atividades de extensão, ja mencionados na \Cref{introduction}, seus orgãos gerenciadores, equipe executora, possíveis processos relacionados, e algumas regras como a de duração mínima de 8 (oito) horas, levando-se em conta o período de organização, execução e elaboração de relatório final.

In \ac{UNIPAMPA}'s view, like all other \acl{HEI}, must have a resolution aimed at standardizing \acp{OA} in general, presenting what they are, their target audience, objectives, etc.
In view of this, \ac{UNIPAMPA}, in CONSUNI/UNIPAMPA Resolution No. 332 of 2021, \cite{Resolucao-332:2021}, determines the types of outreach activities, already mentioned in \Cref{introduction}, its managing bodies, executing team, possible related processes, and some rules such as the minimum duration of 8 (eight) hours, taking into account the period of organization, execution and preparation of the final report.
% ============================================================================

% ============================================================================
% A algum tempo a Unipampa ja vem implantando alguns projetos de extensão dentro de sua grade curricular, por exemplo no curso de Engenharia de Software onde dentro da cadeira de Resolução de Problemas, os alunos se reunem em grupos, semelhante a times de desenvolvimento e gerencia de projetos, sendo eles designados para trabalhar em uma demanda real para alguém da comunidade externa. 
% Esta atividade proporciona ao aluno uma experiência muito recompensadora, pela oportunidade de falar, interagir e contribuir diretamente com um cliente que necessita de ajuda na resolução de algum problema.

For some time now, \ac{UNIPAMPA} has been implementing some outreach projects within its curriculum, for example in the Software Engineering course where, within the Problem Solving subject, students meet in groups, similar to development teams and project management, where they are assigned to work on real demand for someone in the external community.
This activity provides the student with a very rewarding experience, for the opportunity to talk, interact and contribute directly with a customer who needs help in solving a problem.

% ============================================================================

% ============================================================================
% Os principais objetivos na inserção das atividades extensionistas nos cursos de graduação, que a Unipampa ressalta em sua Resolução Nº 317 de 2021, \cite{res317} são os seguintes: 
% \begin{itemize}
%   \item Ajudar ao discente desenvolver sua formação crítica, cidadã, interdisciplinar e responsável;
%   \item Aprimorar como um todo o ensino nos cursos de graduação e fortalecer aa indissociabilidade entre ensino, pesquisa e extensão;
%   \item Fortalecer o compromisso social da Unipampa; 
%   \item Estimular discussões construtivas em todos os setores da Unipampa; 
%   \item Promover ações que fortifiquem os princípios éticos, e o compromisso social da Unipampa em todas as áreas;
%   \item Instigar a comunidade academica a estar mais presente no desenvolvimento humano, academico, social, cultural e economico.
% \end{itemize}

The main objectives in the insertion of outreach activities in undergraduate courses, which \ac{UNIPAMPA} highlights in its Resolution No. 317 of 2021, \cite{res317} are the following:
\begin{itemize}
  \item Help students develop their critical, citizen, interdisciplinary and responsible education;
  \item Improve teaching in undergraduate courses as a whole and strengthen the inseparability among teaching, research and outreach;
  \item Strengthen \ac{UNIPAMPA}'s social commitment;
  \item Stimulate constructive discussions in all sectors of \ac{UNIPAMPA}; 
  \item Promote actions that strengthen \ac{UNIPAMPA}'s ethical principles and social commitment in all areas;
  \item Encourage the academic community to be more present in human, academic, social, cultural, and economic development.
\end{itemize}
% ============================================================================

\subsection{Outreach Programs and Projects}\label{sec:3.2.1}
% Diferença entre eles
% Citar exemplos de Prog. e Proja. dos cursos do campus

% ============================================================================
% Para explicar o que são programas e projetos de extensão, sera utilizado as definições da \textcite{referenciaisPolitica}, este diz que são atividades reguladas internamente pela instituição que articula eventos envolvendo ensino e pesquisa, sempre envolvendo a comunidade externa. 
% Com eles os alunos podem tomar atitudes e decisões diretamente sobre a comunidade em que vive, contribuindo na evolução e progresso da mesma. 
% Além de ajudar a comunidade externa, \ac{FOREXT} diz que os programas e projetos não buscam criar um laço de dependência com a universidade, sendo assim é necessário resolver o problema com mais eficácia e qualidade possível.

To explain what outreach projects and programs are, the definitions of \textcite{referenciaisPolitica} will be used, which says that they are activities regulated internally by the institution that articulates events involving teaching and research, always involving the external community.
With them, students can take attitudes and decisions directly about the community in which they live, contributing to its evolution and progress.
In addition to helping the external community, \ac{FOREXT} says that the programs and projects do not seek to create a bond of dependence with the university, so it is necessary to solve the problem with the most efficiency and quality possible.

% ============================================================================

% ============================================================================
% Pelo fato dos dois termos serem semelhantes, alguma confusão pode acontecer, então \textcite{Viero} destaca a diferença entre os dois, citando as definições feitas pela \ac{ProExt}:

Because the two terms are similar, some confusion can arise, so \textcite{Viero} highlights the difference between the two, citing the definitions made by \ac{ProExt}:
% ============================================================================

% ============================================================================
% \begin{citacao}
% É importante salientar que o ProExt prevê dois conjuntos de ações de extensão universitária: projetos de extensão, definidos como “conjunto de ações processuais contínuas, de caráter educativo, social, cultural ou tecnológico, com objetivo específico e prazo determinado”; e programa de extensão, como “conjunto articulado de projetos e outras ações de extensão, preferencialmente de caráter multidisciplinar e integrado a atividades de pesquisa e de ensino \cite{Viero}
% \end{citacao}

\begin{citacao}
It is important to point out that ProExt provides for two sets of university outreach actions: outreach projects, defined as “a set of continuous procedural actions, of an educational, social, cultural or technological nature, with a specific objective and a determined period”; and outreach program, as “an articulated set of projects and other outreach actions, preferably of a multidisciplinary nature and integrated with research and teaching activities \cite{Viero}.
\end{citacao}
% ============================================================================

% ============================================================================
% Dentro da \ac{UNIPAMPA} campus Alegrete existem alguns projetos e programas vigentes, são exemplos deles com os seus respectivos coordenadores: 
% \begin{inparaenum}[(1)]
%   \item \textbf{Ciência a Cavalo: Universidade e Ensino Básico de Mãos Dadas pelo Fortalecimento da Educação}, Marco Antonio Durlo Tier;
%   \item \textbf{Consultoria de TI para empresas do agronegócio}, Elder de Macedo Rodrigues;
%   \item \textbf{Empresa Júnior: Multi Assessoria e Soluções em Engenharia Júnior - Masé Junior}, José Gabriel Vieira Neto;
%   \item \textbf{Espaço Maker} - Aprendizagem criativa, Vitor Almada;
%   \item \textbf{Programa UniHacker.Club}, Diego Luiz Kreutz;
%   \item \textbf{UNIPATAS Alegrete: Proteção, Esterilização e Adoção}, Camila da Costa Lacerda Tolio Richardt;
%   \item \textbf{Programa C}, Aline Vieira de Mello;
%   \item \textbf{Programa JEDI}, Maicon Bernardino da Silveira.
% \end{inparaenum}

% ########################
% Professor, fiquei em dúvida se traduzia os nomes dos programas de extensão, dai eu traduzi so o que eu achei que não fazia parte do nome em si.
% ########################
Within the \ac{UNIPAMPA} Alegrete campus there are some current projects and programs, examples of which are with their respective coordinators:
\begin{inparaenum}[(1)]
  \item \textbf{Ciência a Cavalo: University and Basic Education Hand in Hand for Strengthening Education}, Profº Marco Antonio Durlo Tier;
  \item \textbf{IT consultancy for Agribusiness Companies}, Profº Elder de Macedo Rodrigues;
  \item \textbf{Empresa Júnior: Multi Advisory and Solutions in Junior Engineering - MASE Junior}, Profº José Gabriel Vieira Neto;
  \item \textbf{Espaço Maker} - Criative Learning, TAE Vitor Almada;
  \item \textbf{Programa UniHacker.Club}, Profº Diego Luiz Kreutz;
  \item \textbf{UNIPATAS Alegrete: Protection, Sterilization and Adoption}, TAE Camila da Costa Lacerda Tolio Richardt;
  \item \textbf{Programa C}, Profª Aline Vieira de Mello;
  \item \textbf{Programa JEDI}, Profº Maicon Bernardino da Silveira.
\end{inparaenum}
% ============================================================================

% ============================================================================
% Utilizando a página online do ultimo programa citado \textcite{JEDI}, este é um programa que se propoe a resolver problemas locais utilizando tecnologia e envolvimento com a comunidade. 
% No primeiro ciclo do programa quatro projetos de extensão foram propostos, cada um com seus objetivos, metodologias e atividades próprios, são eles:
% \begin{inparaenum}[(1)]
%   \item Padawan Academy;
%   \item Jedi Apprentice;
%   \item Jedi Problem-Solving;
%   \item Jedi Mind.
% \end{inparaenum}

% Using the online page of the last mentioned program \textcite{JEDI}, this is a program that proposes to solve local problems using technology and community involvement.
% In the first cycle of the program, four outreach projects were proposed, each with its own objectives, methodologies and activities, they are:
% \begin{inparaenum}[(1)]
%   \item Padawan Academy;
%   \item Jedi Apprentice;
%   \item Jedi Problem-Solving;
%   \item Jedi Mind.
% \end{inparaenum}

% ============================================================================

\subsection{Processes for New Proposals for Outreach Programs and Projects}

% ============================================================================
% Para ser realizado o novo cadastro de um programa ou projeto de extensão e sua geração de certificados ao final, existem algumas normas definidas pela \textcite{Resolucao-332:2021}, que devem ser desempenhados antes. 
% Tendo em maos estes documentos a \ac{UNIPAMPA} normatizou alguns esquemas de fluxo dos processos, para que todos os proponentes tenham conhecimento do que acontece depois que a proposta é realizada.

In order to register a new outreach program or project and generate certificates at the end, there are some rules defined by \textcite{Resolucao-332:2021}, which must be performed beforehand.
With these documents in hand, \ac{UNIPAMPA} standardized some process flow schemes, so that all proponents are aware of what happens after the proposal is made.
% ============================================================================

% ============================================================================
% Na \Cref{fig:outreach-projects-registration}, o fluxo de cadastro de um novo projeto de extensão é apresentado, nele é possivel ver que a proposta passa por diversos passos de correções e avaliações, sendo enviada para diversos atores ao longo do processo.
% Por fim sendo decisão da \ac{PROEXT} solicitar alterações finais, ou homologar o projeto, deferindo um novo número de registro.

In \Cref{fig:outreach-projects-registration}, the registration flow of a new outreach project is presented, in which it is possible to see that the proposal goes through several steps of corrections and evaluations, being sent to several actors throughout the process.
% ============================================================================

\begin{figure}[htb]
  \caption{Outreach Projects Registration}\label{fig:outreach-projects-registration}
  \begin{center}
    \includegraphics[width=16cm]{img/3-registro-de-projetos-de-extensao.png}
  \end{center}
  \fonte{Adapted from \cite{siteProcessos}.}
\end{figure}

% ============================================================================
% Já na \Cref{fig:issuance-certificates}, estão representados os passos relacionados a homologação e geração de certificados, comecando com o proponente da atividade tendo em maõs a lista de presença e a planilha com informações para a geração dos certificados, logo um relatorio final é construido e inserido no sistema \ac{SIPPEE}, este é avaliado e homologado, chegando novamente até a \ac{PROEXT} que com a planilha enviada, envia seus dados para o sistema \ac{SGCE}, recebendo os certificados e os enviando para os emails dos participantes. 

In \Cref{fig:issuance-certificates}, the steps related to the approval and generation of certificates are represented, starting with the proponent of the activity having the attendance list and the spreadsheet with information for the generation of certificates. 
Then, a final report is built and inserted in the \ac{SIPPEE} system, it is evaluated and approved, reaching again at \ac{PROEXT} which, with the spreadsheet sent, sends its data to the \ac{SGCE} system, receiving the certificates and sending them to participants' emails.
% ============================================================================

\begin{figure}[htb]
  \caption{Issuance of certificates}\label{fig:issuance-certificates}
  \begin{center}
    \includegraphics[width=16cm]{img/3-emissao-de-certificados.png}
  \end{center}
  \fonte{Adapted from \cite{siteProcessos}.}
\end{figure}

\subsection{``Unipampa Cidadã'' Program}\label{sec:3.2.2}
% https://unipampa.edu.br/portal/sites/default/files/documentos/instrucao_normativa_18-2021_revoga_in-17-2021_normatiza_o_programa_institucional_unipampa_cidada.pdf
% INSTRUÇÃO NORMATIVA UNIPAMPA Nº 18, 05 DE AGOSTO DE 2021

% ============================================================================
% A \ac{UNIPAMPA} atravez da Instrução Normativa Nº 18 \cite{unipampacidada}, usando a Resolução Nº317 \cite{res317}, foi estabelecido que o projeto de extensão chamado "UNIPAMPA Cidadã" deverá ser ofertado por todos os cursos, sendo composto por atividades de cidadania e solidariedade e com o objetivo de  formar egressos cientes de sua responsabilidade social, estimulando e aumentando a integração com a comunidade local.

\ac{UNIPAMPA} through Normative Instruction No. 18 \cite{unipampacidada}, using Resolution No.317 \cite{res317}, established that the outreach project called ``Unipampa Cidadã'' must be offered by all courses, consisting of citizenship and solidarity activities and with the objective of training graduates aware of their social responsibility, stimulating and increasing integration with the local community.
% ============================================================================

% ============================================================================
% Após a implementação do projeto nos cursos da instituição, esta deverá ser realizada por todos os discentes, a cadeira ofertada para o projeto deverá ter no mínimo 60 e no máximo 120 horas. 
% As ações comunitárias devem ser realizadas em instituições públicas, \acp{NGO} e organizações ou associações da sociedade civil organizada. O supervisor de extensão do curso é o encarregado por fazer a avaliação do projeto, planejamento, acompanhamento, validação e ele será responável por aprovar o inicio das atividades.

After the implementation of the project in the institution's courses, it must be carried out by all students, the course offered for the project must have a minimum of 60 and a maximum of 120 hours.
Community actions must be carried out in public institutions, \acp{NGO} and organizations or associations of organized civil society. 
The course outreach supervisor is responsible for carrying out the project evaluation, planning, monitoring, validation and he will be responsible for approving the beginning of the activities.
% ============================================================================

% ============================================================================
% O projeto também disponibiliza na Instrução Normativa Nº 18, um modelo de formulário para preenchimento de dados quando as atividades são finalizadas, permitindo o discente refletir sobre o impacto do projeto sob sua visão apontando seus aprendizados durante a execução. 
% Por fim o supervisor pode realizar observaçoes sobre o discente e indicar se este foi aprovado ou reprovado.

The project also makes available in Normative Instruction Nº 18, a form template for filling in data when the activities are completed, allowing the student to reflect on the impact of the project under their view, pointing out what they learned during the execution.
Finally, the supervisor can make observations about the student and indicate whether he or she passed or failed.
% ============================================================================
\section{Similar Outreach Support Tools}\label{sec:3.3}
% Overview de soluções
% Descrever em alto nível e fazer gancho (spoiler) com o Capítulo do grey
% Funcionalidades comuns
% Foi feito um detalhamento metodologico no capitulo 4.1
% Tentamos fazer revisão sistematica, encontramos duas mas preferimos a cinza
% Citar artigos que falam do ferramental

% ============================================================================
% Em conjunto com o \Cref{cap:grey} que ira apresentar a revisão conduzida na literatura cinza, algumas ferramentas foram pesquisadas para adiquirir informaçoes de como o mercado está em relaçao a extensão nas universidades. 
% Com os resultados foi possível levantar funcionalidades, detalhes e pontos em comum dentre as ferramentas.

In conjunction with \Cref{grey_literature} that will present the review conducted in the grey literature, some tools were researched to acquire information on how the market is in relation to outreach in universities.
With the results it was possible to raise functionalities, details and common points among the tools.
% ============================================================================

% ============================================================================
% Em um primeiro momento os autores buscaram fazer uma revisão sistematica na literatura branca, mas os resultados encontrados não satisfazeriam por completo, visto que a exploração manual por várias ferramentas relacionadas ao tema, traria mais conteúdo para ser classificado e discutido entre os envolvidos na pesquisa.

At first, the authors sought to make a systematic review of the white literature, but the results found would not be completely satisfactory, since the manual exploration by various tools related to the topic would bring more content to be classified and discussed among those involved in the research.
% ============================================================================

% ============================================================================
% Durante a execução da revisão, a ferramenta que mais retornou resultados e estava sempre presente nas pesquisas, foi a \ac{SIGAA}, a qual é a mais utilizada por diversas instituições, sendo ela muito completa contendo partes em seu sistema voltados para a maioria dos processos que envolvem uma instituição. 
% Outra que apresentou resultados interessantes foi a \ac{CAEX}, que apresentou diversas funcionalidades únicas, sendo apenas ela que as apresentava, com ela foi possível retirar ideias de grande importancia para a construção de uma ferramenta completa.

During the execution of the review, the tool that returned the most results and was always present in the research was \ac{SIGAA}, which is the most used by several institutions, being very complete, containing parts in its system aimed at most processes involving an institution.
Another one that presented interesting results was \ac{CAEX}, which presented several unique features, being only it that presented them, with this it was possible to extract ideas of great importance for the construction of a complete tool.
% ============================================================================
\section{Chapter Summary}\label{sec:3.4}
% ============================================================================
% Neste capítulo foi apresentado diretrizes de varias resoluçoes e normativas relacionadas a extensão, tanto no pais como um todo, quanto na \ac{UNIPAMPA}. 
% Também foi discutido sobre as semelhanças e diferenças entre os termos programa e projeto de extensão, apresentando os processos mais relevantes envolvidos no seu periodo de vida.
% Como exemplo mais recente de programa de extensão, a "UNIPAMPA Cidadã" teve parte de seus objetivos e diretrizes apresentados, por fim foi discutido um pouco sobre a revisão na literatura cinza executada pelos participantes desta pesquisa, logo no proximo capítulo será mais aprofundado critérios, metodologia, resultados, questões de pesquisa, dentre outras informaçoes pertinentes a literatura cinza.

In this chapter, guidelines of various resolutions and regulations related to outreach were presented, both in the country as a whole and in \ac{UNIPAMPA}.
It was also discussed the similarities and differences between the terms outreach program and project, presenting the most relevant processes involved in its life span.
As a more recent example of an outreach program, ``Unipampa Cidadã'' had part of its objectives and guidelines presented, finally, a little discussion about the grey literature review carried out by the participants of this research was discussed, so in the next chapter, criteria will be more in-depth, methodology, results, research questions, among other information relevant to grey literature.
% ============================================================================   % 20/07
%==============================================================================
\chapter{GREY LITERATURE}\label{grey_literature}
%==============================================================================

Before beginning to develop the solution itself, it would be extremely beneficial to conduct a systematic review of the grey literature to map and assess existing tools and solutions that already address the issue of managing outreach activities in the context of \acp{HEI}. This research will ultimately result in a software product. Two authors did the review. Although the two term papers were prepared independently, as was already indicated, the artifacts produced to support the study were produced jointly.

The systematic review of the gray literature is described in this chapter. Additionally, data gathered during the study that is pertinent to the creation of the target product will be presented. In addition to a thorough examination and comparison of the chosen tools, the protocol established to conduct the evaluation will be covered, citing details such research questions, inclusion and exclusion criteria, extracted data, and search strings.

In this manner, the chapter is structured: Introduced in the \Cref{sec:gl-background} are words and ideas utilized in the study. The technique outlined by the authors will be presented in \Cref{sec:gl-planning}. The methods used in the study and the information gathered to address the research questions will be explained in the section on reporting \Cref{sec:gl-reporting}, while the section on validity \Cref{sec:gl-validity} highlights risks to the study's validity. The systematic review is concluded by \Cref{sec:gl-considerations}.

\section{Background}\label{sec:gl-background}

The following definition of grey literature comes from \citeonline{garousi2019guidelines}:
\begin{citacao}
  <grey literature> is produced at all levels of government, academia, business, and industry in print and electronic formats, but is not controlled by commercial publishers, or that is, where publication is not the main activity of the producing body.
\end{citacao}

The quality of software described as a ``black box'' is one in which the internal workings of the system are unknown; its use solely concentrates on the outputs produced in response to chosen inputs and execution conditions \citeonline{nidhra2012black}.

This phrase was used in relation to the Google search engine, where it is unknown exactly what occurs internally other than the fact that occasionally, despite the identical search word, the results differ just little.

\section{Planning}\label{sec:gl-planning}

Due to the limited amount of formal works published on the issue of outreach activities management, the authors determined that a systematic review of the grey literature would be more interesting and valuable to the study than one in the white literature.

\subsection{Reasons for Carrying out the Review}\label{sec:gl-planning-motives}

The following were the key justifications given by the authors to include a review of grey literature in their study:
\begin{inparaenum}[(i)]
    \item More tools than formal articles in search results;
    \item Very few results were obtained when the search terms were applied to white literature;
  \item There are a number of tools and solutions without published articles;
  \item The authors are looking for tools in order to gather inspiration and useful design ideas for the creation of the intended product.
\end{inparaenum}

The questions and their responses that were used to make the choice to conduct the review of the grey literature can be found in the \Cref{tab:questoesgarousi}. Additionally, the following objectives were specified for carrying out the review:

\begin{inparaenum}[(i)]
  \item Find free tools that help academic management in some way;
  \item Look for features in tools already in existence and
  \item Validate concepts for the features and information that will be used in the solution.
\end{inparaenum}

\input{componentes/4-research-questions-garousi}

\subsection{Research Questions}\label{sec:gl-planning-rq}

The research questions that the authors have identified for the systematic review are listed in the \Cref{tab:research-questions}.

\input{componentes/4-defined-rq.tex}

The search terms were developed by modifying the approach utilized in \cite{godin2015applying}. The first step was to establish search phrases using words like \textbf{extensão} (outreach), \textbf{programa} (program), \textbf{projeto} (project), \textbf{gerenciamento} (management) and \textbf{atividade} (activity).

Additionally, because the search engine's site filter was initially employed and the scope of the project was restricted to outreach initiatives at Brazilian universities, only websites with the specified ``.edu.br'' ending would be displayed. Later on, it was discovered that it would have been wiser to remove the filter because some private universities do not use the .edu domain extension.

In the end, the authors generated ten search strings, seven of which combined the terms ``\textbf{extensão} (\textbf{programa} | \textbf{projeto})'', which were deemed to be the most pertinent terms. There were 100 entries per string and a limit of only using the first ten pages of the search engine's results meant that there were a total of 1000 records.

After the initial search, the keyword \acs{SIGAA} was eliminated because it is a resource used by many public universities called \citeonline{das2013sistema}, which clogged the results with virtually the same record and would have concealed other alternatives. In \Cref{tab:gl-strings}, the defined strings are displayed.

\input{componentes/4-search-strings.tex}

The Google search engine was used to conduct the actual search for the strings.

\subsection{Inclusion Criteria}\label{sec:gl-planning-inc}

The inclusion criteria were developed over the course of two stages. The authors implemented a filter in the first stage to distinguish tools from catalogs due to the significant number of institutional sites that were simply catalogs of outreach initiatives. The outcome must meet at least three of the following standards in order to be considered:
\begin{inparaenum}[(a)]
  \item User login;
  \item Registration of activities;
  \item Activity listing;
  \item Possibility of signing up for outreach activities.
\end{inparaenum}

Step 2 was implemented once the results had been filtered using the aforementioned criteria. It had stricter definitions of what was required to be included. They are listed in \Cref{tbl:gl-inclusion-criteria} as follows:

\input{componentes/4-inclusion-criteria.tex}

\subsection{Exclusion Criteria}\label{sec:gl-planning-exc}

Exclusion criteria were also established, and any result that met even one of these was automatically disqualified from further consideration. Six criteria were initially created by the authors, but following alignments with the adviser, it was determined that two of them were superfluous. The remaining factors, which affected the results, are shown in \Cref{tbl:gl-exclusion-criteria}.

\input{componentes/4-exclusion-criteria.tex}

\subsection{Quality Criteria}\label{sec:gl-planning-qlty}

Five quality criteria that are focused on traits deemed relevant within a tool and how it differs from the others were created to evaluate the quality of the tools that passed the inclusion and exclusion criteria. The scale used in the article by \citeonline{iung2020systematic} was modified to quantify the scores for each criterion and is as follows:
\begin{inparaenum}[(i)]
  \item \textbf{Y}es: 1.0;
  \item \textbf{P}artially: 0.5;
  \item \textbf{N}o: 0.
\end{inparaenum}
The defined criteria are shown in \Cref{tbl:gl-quality-criteria}.

\input{componentes/4-quality-criteria.tex}

\subsection{Data Extraction Strategy}\label{sec:gl-planning-datastrategy}

After the final list of tools is chosen, a manual data extraction is done in order to respond to the research questions that have been established \Cref{tab:research-questions}. In the beginning, we look for all the \ac{OA} functionalities the program has, creating a data matrix. There is a list of all the various capabilities that were discovered within the findings. The matrix is discussed in more detail in the \Cref{gl-feature-matrix} below.

Afterwards, a new manual extraction was carried out while highlighting the first four most pertinent properties that were shared by all of the studied tools. Now with the intention of discovering every feature these solutions possessed. It is much simpler to handle comparable issues that will ultimately arise when constructing the goal product if this data is refined and tabulated.

\section{Reporting}\label{sec:gl-reporting}

With the goal of starting and terminating on days that were close together, the search and record mapping were conducted between February 17 and February 20, 2022, decreasing one of the dangers to validity.

\subsection{Research}\label{sec:gl-research}

Both authors contributed equally to the overall workload. In this manner, each person examined five of the ten pages using the search term, yielding fifty results per search string and 500 results per author. The first set of results, as displayed in \Cref{tbl:gl-search-results}, consisted of 169.

There were 56 results left after applying the first step of the inclusion criterion. The findings were then further decreased once the verification with the second step of the inclusion and exclusion criteria was completed, with 19 tools failing \textbf{IC 1.}, 8 tools failing \textbf{IC 2.}, and 24 tools being rejected for failing \textbf{IC 3.} Regarding the exclusion criterion, only one tool was eliminated by \textbf{ECs 1. and 2.}, as well. However, 14 tools failed \textbf{EC 3.}, and the same number failed \textbf{EC 4.} As can be seen in \Cref{fig:gl-results-criteria}, there were only 12 tools and websites left to be examined.

\input{componentes/4-search-results.tex}

\begin{figure}[!htb]
  \caption{Results x Criteria}\label{fig:gl-results-criteria}
  \begin{center}
    \includegraphics[width=12cm]{img/4-results.png}
  \end{center}
  \fonte{Author.}
\end{figure}

\subsection{Data Extraction}\label{sec:gl-data-extraction}

This section explains how the two data extractions from the discovered tools were carried out: one for the feature matrix and the other to collect more details on the key features shared by the tools.

\subsubsection{Feature Matrix}\label{sec:gl-feature-matrix}

It was important to develop a functions matrix among the filtered results after the research was completed in order to apply the quality standards. The authors were able to determine which features were present in the examined tools the most frequently in this method. There were determined to be 37 traits in total, some of which repeated more frequently than others. The matrix can be seen in \Cref{fig:gl-matrix}.

\begin{figure}[!htb]
  \caption{Feature Matrix}\label{fig:gl-matrix}
  \begin{center}
    \includegraphics[width=16cm]{img/4-functionality-matrix.pdf}
  \end{center}
  \fonte{Author.}
\end{figure}

Lighter gray highlights were utilized to draw attention to the characteristics that were shared by all of the examined tools and websites so that they could be used as criteria in the subsequent stage of data extraction.

\subsubsection{More Information from Important Features}\label{sec:gl-data-extraction-2}

The goal of the second data extraction was to determine which data was utilized to 
\begin{inparaenum}[(i)]
  \item Listing of outreach activities;
  \item Detailed page of an activity;
  \item Enrollment of a participant into an activity;
  \item Registration of users external to the institution.
\end{inparaenum}

It was challenging to unify the analysis because each tool has its own format and attribute naming, thus the original names were retained. To prevent confusion, tools that lacked the chosen features have been highlighted in grey rather than having the cells left blank. Because it was nearly impossible to try to follow a pattern for all the tools, the extracted findings are written informally. The extracted data can be seen in \Cref{gl-additional-extraction}.

\begin{figure}[!htb]
  \caption{Additional Information Extraction}\label{fig:gl-additional-extraction}
  \begin{center}
    \includegraphics[width=16cm]{img/4-data-extraction-2.pdf}
  \end{center}
  \fonte{Author.}
\end{figure}

\subsection{Tool Classification}\label{sec:gl-tool-classification}

The extracted and tabulated data allowed for the classification of the tools based on the previously established quality standards. The scoring range for a tool is from 0 (zero) to 5 (five). The final results are shown in the table \Cref{tbl:gl-quality-criteria-results}.

\input{componentes/4-quality-criteria-results.tex}

With this classification, it is clear that the \ac{CAEX} tool and \ac{SIGAA} received the highest ratings, which was exactly what was anticipated. First, \ac{SIGAA} is one of the academic management tools that institutions in the nation utilize the most, and \ac{CAEX} is the tool that offered the most distinctive features. As a result, they were two instruments that had a lot of promise and were very helpful in gathering data to create the goal product.

\subsection{Answering the Research Questions}\label{sec:gl-answer-research-questions}

The research questions are under the \Cref{tab:research-questions} and were introduced earlier in the study. For convenience's sake, each question is also explained below.

\begin{itemize}
  \item \textbf{RQ 1.} What tools currently exist that perform academic management?

        This is a question that also refers to some instruments that were eliminated during the use of inclusion and exclusion criteria. In this instance, 36 tools supporting academic management of various kinds were found, but only 12 of them meet the required requirements and are mentioned in the tool matrix in \Cref{fig:gl-matrix}.
  \item \textbf{RQ 1.1.} Which ones have related functionality or support outreach activities?

        The following tools were found, as it was already demonstrated in the \Cref{fig:gl-matrix}, which describes the relationships between tools and features:
        \begin{inparaenum}[(1)]
          \item Cachalote;
          \item CAEX;
          \item Einstein;
          \item ENS;
          \item Santa Marcelina;
          \item SGE;
          \item SIEX;
          \item SIG;
          \item SIGAA;
          \item SUAP;
          \item UNINASSAU and
          \item UNINTER.
        \end{inparaenum}

  \item \textbf{RQ 1.2.} What are the features offered by these tools?

        The features matrix, which is present in \Cref{fig:gl-matrix} and has a total of 37 features, contains a list of every feature that was discovered.
  \item \textbf{RQ 1.3.} What are the most common features between this type of tool?

        The most common functionalities in this type of tool are:
        \begin{inparaenum}[(i)]
          \item A login system;
          \item Lististing of \aclp{OA};
          \item \ac{OA} details page;
          \item \ac{OA} enrollment and
          \item Registration of external users.
        \end{inparaenum}
        The ability to search for events by text is another feature that is present regularly but not as frequently as the other features; 8 of the tools were found to support this functionality.
  \item \textbf{RQ 1.4.} What data do the tools use in relation to activities, participant registration and user registration?

        By analyzing the second data extraction presented in \Cref{sec:gl-data-extraction-2}, the most common fields for \acp{OA} are:
        \begin{inparaenum}[(a)]
          \item Title;
          \item Duration;
          \item Enrollment period;
          \item Contact information;
          \item Description;
          \item Target audience;
          \item Faculty and
          \item Schedule.
        \end{inparaenum}

        Regarding enrollment, the most common fields found are:
        \begin{inparaenum}[(a)]
          \item Participant's personal data;
          \item Institutional affiliation;
          \item Participant type and
          \item Information about the participant's disability, if any.
        \end{inparaenum}

        When it comes to user registration, these tools mostly employ personal information, authentication information, and an address; however, some also request information about the institution, participant type, and professional data.
\end{itemize}

\section{Validity}\label{sec:gl-validity}

Some validity threats were found as the systematic review mapping process progressed. While the writers were able to lessen the majority of them, some still need to be addressed. They are as follows:

\begin{itemize}
  \item When comparing the findings they both found throughout the study phase, the authors observed that the search results varied just little, between one or two different records. Although it was a hazard that could be easily reduced, it couldn't be fully ruled out. In order to do the search in anonymous mode, the approach employed was to log out of the account currently logged into the browser. As a result, there were fewer divergences overall, however occasionally divergent outcomes did occur.
  \item Functionalities of the tools weren't checked with the creators. Unfortunately, the authors were unable to reach any universities to inquire about the management solution being employed.
  \item The authors aimed to conduct the search in as little time as feasible, beginning and finishing it in just three days, in order to reduce the divergence of findings. As the search engine is regarded as a ``black box'', making it challenging to predict the precise results that will emerge with each search string, the longer the delay, the greater the chance that risks to the study will be introduced.
\end{itemize}

\section{Considerations}\label{sec:gl-considerations}

Finding tools that are similar to the intended outcome of the entire study was made possible by this thorough review of the grey literature. Before undertaking the review, no knowledge of the current status of the field or the most popular solutions employed by Brazilian \ac{HEI} existed.

A wealth of useful data was gathered regarding the instruments used today. It was now much more obvious what \aclp{OA} management and processes covered. This information will be useful when putting the objective product into use, which seeks to provide a comprehensive solution for \ac{OA} management.

Furthermore, the review protocol's definitions of all the research questions could all be answered.         % 27/07
%==============================================================================
\chapter{SURVEY}\label{sec:5}
%==============================================================================

%==============================================================================
Neste capítulo informações mais detalhadas são apresentadas sobre o levantamento (\textit{survey}) que foi conduzido. 
Na \Cref{sec:survey-protocol}, é apresentado detalhes sobre o protocolo adotado, autor de referencia e divisão de tarefas entre os pesquisadores. 
Logo na \Cref{sec:survey-threats}, as ameaças a validade do estudo são relatadas, e por fim na \Cref{sec:survey-results}, todos os resultados alcançados durante a execução são discorridos.

%==============================================================================

\section{Survey Protocol} \label{sec:survey-protocol}

%==============================================================================
Um levantamento (\textit{survey}) é uma abordagem de coleta e análise de dados em que os participantes respondem a perguntas ou a declarações que foram desenvolvidas antecipadamente. 
O protocolo escolhido para a elaboração desta pesquisa foi inspirado nas diretrizes propostas por \citeonline{kasunic2005designing}, em \textit{Designing an effective survey} e está ilustrado na \Cref{fig:setepassos}.

%==============================================================================

\begin{figure}[htb]
  \caption{Seven steps of the research process}\label{fig:setepassos}
  \begin{center}
    \includegraphics[width=16cm]{img/kasunic_process.png}
  \end{center}
  \fonte{\cite{kasunic2005designing}.}
\end{figure}

%==============================================================================
Como será dito posteriormente, o objetivo é compreender as necessidades de discentes e docentes em relação aos projetos e atividades de extensão. 
A escolha do \textit{survey} como abordagem de coleta de dados se deve ao fato de que as características de uma pesquisa deste tipo nos permite generalizar sobre as crenças e opiniões de muitas pessoas estudando apenas um subconjunto delas \cite{kasunic2005designing}. 
Sendo, neste caso, a ferramenta ideal.

%==============================================================================

%==============================================================================
Tendo em vista que esta pesquisa foi executada por dois estudantes, a carga de trabalho foi dividida, de maneira que a qualidade e desempenho fossem melhorados. 
Na \Cref{tbl:survey-tasks} se encontra a divisão de atividades adotada, contemplando as já definidas por \citeonline{kasunic2005designing}.

%==============================================================================

% Tabela \Cref{tbl:survey-tasks} Activity Division
\input{componentes/survey}

\subsection{Identify the Research Objectives} \label{sec:survey-objectives}
% Objetivo do survey -> refinamento dos requisitos, validar, importancia
% Transformar em uma questão de pesquisa, pergunta
% grey gerou resultados pra usar no survey com o olhar dos futuros usuarios

%==============================================================================
O objetivo deste primeiro passo é identificar qual a importância e o por que de fazer um survey, o que poderia ser conquistado com ele.
Levando em conta os resultados gerados pela revisão na literatura cinza, mencionados no \Cref{grey_literature}, foi possível elaborar questões de maneira que o participante informe, na sua visão, a importância de determinado requisito levantado. 
Logo, o objetivo deste survey é ordená-los por prioridade, utilizando a opinião de possíveis usuários finais.
%==============================================================================

%==============================================================================
Além de ser perguntado a opinião dos participantes, foi permitido com que eles fornecessem sugestões ou melhorias em relação a requisitos da ferramenta, já que um dos objetivos da pesquisa está voltado a entender as necessidades dos possíveis usuários do sistema. 
Assim, tendo uma base mais sólida para começar o processo de desenvolvimento da solução corretamente, com os escopos das atividades mais bem definidos.

%==============================================================================
\subsection{Identify and Characterize the Target Audience} \label{sec:survey-targets}
%==============================================================================
Neste estágio, é necessário olhar para os possíveis públicos respondentes e identificar quem será o público respondente e quem é a população do estudo. Assim sendo, a população é composta por todos as pessoas dentro da comunidade acadêmica, logo foi escolhido para representarem a amostra desta população os coordenadores de programas ou projetos de extensão, docentes e discentes, tendo preferência em participantes que tenham experiência com atividades de extensão. Com este público é possível ter o ponto de vista de todos os usuários da ferramenta, quem cria atividades e quem se inscreve em uma.
%==============================================================================
% Para possívelmente conseguir melhores sugestões nas respostas do survey, foi necessário abrangir a maior quantidade de campus da \ac{UNIPAMPA} possivel, era esperado que campus como Uruguaiana, Bagé e Dom Pedrito que, como visto na \Cref{fig:number-of-projects} foram os campus que mais possuiram atividades de extensão no ano de 2017 de acordo com \cite{relatorio-2017}, fornecessem mais respondentes, mas no final isto nao aconteceu.

% \begin{figure}[htb]
%   \caption{Number of Projects Contemplated in the Internal Public Notices}\label{fig:number-of-projects}
%   \begin{center}
%     \includegraphics[width=16cm]{img/uruguaiana.pdf}
%   \end{center}
%   \fonte{Adapted from \cite{relatorio-2017}}
% \end{figure}

\subsection{Design the Sampling Plan} \label{sec:survey-sampling}

%==============================================================================
De acordo com \citeonline{kasunic2005designing}, o objetivo desta fase é determinar os seguintes tópicos:
\begin{itemize}
    \item How individuals will be selected to participate in the survey;
    \item The required size of the sample.
\end{itemize}
%==============================================================================

%==============================================================================
Por isso, o primeiro tópico buscou abrangir a maioria de campus da \ac{UNIPAMPA} possível por meio de envio de emails para as suas secretarias academicas, direcionados para os alunos e para listas de coordenadores de programas e projetos de extensão, mantendo o equilibrio entre docentes e discentes. 
Com isso, campus como Uruguaiana, Bagé e Dom Pedrito que, como visto na \Cref{fig:number-of-projects} foram os campus que mais possuíram atividades de extensão no ano de 2017 \cite{relatorio-2017}, assim, esperava-se que fornecessem mais respondentes para a pesquisa.
%==============================================================================
\begin{figure}[htb]
  \caption{Number of Projects Contemplated in the Internal Public Notices}\label{fig:number-of-projects}
  \begin{center}
    \includegraphics[width=16cm]{img/uruguaiana.pdf}
  \end{center}
  \fonte{Adapted from \cite{relatorio-2017}}
\end{figure}

%==============================================================================
Em relaçao ao tamanho mínimo da amostra, tamanhos entre 100 e 150 respondentes já seriam suficientes, pois além das respostas quantitativas teriam todas as respostas qualitativas com sugestões e melhorias, demandando mais tempo para análise.
%==============================================================================

%==============================================================================
A separação da amostra é um ponto essencial para a melhor eficiência do survey, estando de acordo com a prática recomendada 22 definida por \citeonline{Jefferson}, em que diz que a amostra deve ser dividida de acordo com as suas caracteristicas e semelhanças. 
Para contemplá-la, os respondentes do questionário que se declarassem como \acp{TAE} ou docentes eram direcionados para uma área do questionário, e discentes para outra, ambas as áreas com perguntas relacionadas ao perfil reclarado pelo respondente.

%==============================================================================
\subsection{Design and Write the Questionnaire} \label{sec:survey-questionnaire}

% (kasunic) objetivos e caracteristicas da sample para fazer o questionario, importancia dos dois
% (guideline) resultados dependem da qualidade das perguntas

% (kasunic) 4 tipos de perguntas, colcoar as que se encaixam em cada uma
% (kasunic) close ended e open ended questions 52, todas sao ordinal responses usando MoSCoW

% Colocar o questionario em apendice
%==============================================================================
\citeonline{kasunic2005designing} ressalta que para a estruturação e escrita do questionário, os objetivos de pesquisa e as características da amostra devem ser levados em conta. 
De acordo com o autor, questionarios que não possuem objetivos bem definidos tem mais chances de possuirem perguntas que só consomem tempo do respondente, ele ressalta isso com uma pergunta \citeonline[p.34]{kasunic2005designing} ``How can you reach insightful conclusions if you do not know what you were looking for or planning to observe?'', neste questionário o objetivo é bem definido, focado em priorização de requisitos e levantamento de sugestões pelos possíveis usuários finais como bem descrito na \Cref{sec:survey-objectives}. 
Da mesma maneira, as caracteristicas da amostra são importantes para escrever as perguntas de um modo que todos entendam e não apenas pensando no entendimento dos proprios pesquisadores. 
% \citeonline[p.23]
\citeonline{surveyGuidelines} atenta que os resultados que serão obtidos com o survey, estão diretamente relacionados com a qualidade do questionário utilizado.
%==============================================================================

%==============================================================================
Para \citeonline{surveyGuidelines} existem dois tipos de questionários, self-administrated and interviewer-administrated questionnaire, de acordo com suas definições este se encaixa no primeiro tipo, pois por ser um questionario web-based, não é necessário ao acompanhamento dos pesquisadores. 
Este modelo permite a maior abrangência de respondentes, mas por outro lado tende a maior taxa de desistência, ressaltando a importância de uma boa estruturação.
%==============================================================================

%==============================================================================
Para a realização do survey, foi escolhida a ferramenta do Google Forms, já que ela contribui com uma interface simples e de facil entendimento, logo que esta já é utilizada pela maioria dos perfis dos respondentes.
%==============================================================================

%==============================================================================
A estrutura do questionário que está contido no \Cref{annex:questionnaire} se da pela página inicial, questões de perfil do respondente, questões de priorização de requisitos e por fim sugestões de funcionalidades, estas estão descritas a seguir em suas respectivas seções.
%==============================================================================
\subsubsection{The Welcome Screen}
%==============================================================================
% \citeonline[p.65]
Seguindo instruções de \citeonline{kasunic2005designing}, a primeira página do questionário contém informações importantes para o participante como:

%==============================================================================

%==============================================================================
\begin{itemize}
  \item O objetivo da pesquisa;
  \item Duração estimada do questionário;
  \item Endereços de email para contato;
  \item Pesquisadores envolvidos;
  \item Caráter voluntário, anônimo e confidencial da pesquisa;
  \item Instituição e organização envolvida.
\end{itemize} 
Por fim perguntando para o participante se ele aceita em continuar com a pesquisa.

%==============================================================================

\subsubsection{Profile Questions}\label{survey:profile-questions}
%==============================================================================
As questões referentes a adiquirir informações sobre o participante são importantes nas primeiras fases do questionario, pois motivam os participantes a continuar respondendo-o sem confundi-los com perguntas complexas logo no começo, \cite{LMRea}. Além que com uma boa classificação de participantes, permite que a analise destes seja feita de maneira mais controlada e organizada como bem mencionado por \citeonline{legramante}.
%==============================================================================

%==============================================================================
Os dados que foram retirados com as perguntas de perfil, são listadas a seguir:
\begin{inparaenum}[(1)]
  \item Se o participante faz parte da \ac{UNIPAMPA};
  \item Sexo do participante;
  \item Faixa etária;
  \item Formação acadêmica;
  \item Se o participante ja esteve em alguma atividade extensionista;
  \item Se a anterior for verdadeira, quais papeis ele desempenhou;
  \item Papel do participante na comunidade acadêmica;
  \item Campus/Cidade do participante;
  \item Curso em que esta relacionado.
\end{inparaenum}
%==============================================================================

\subsubsection{Requisites Priorization Questions}
%==============================================================================
Nas perguntas relacionadas ao objetivo da pesquisa foram utilizados alguns direcionamentos descritos por \citeonline{forza}, são eles:
% \citeonline[p.168]
\begin{description}
  \item \textbf{Suggestion 1.} Define the way questions are asked to collect the information on a specific concept;\label{suggestion:1}
  \item \textbf{Suggestion 2.} For each question decide the scale on which the answers are placed;\label{suggestion:2}
  \item \textbf{Suggestion 3.} Identify the appropriate respondent(s) to each question;\label{suggestion:3}
  \item \textbf{Suggestion 4.} Put together the questions in questionnaires that facilitate and motivate the respondent(s) to respond.\label{suggestion:4}
\end{description}
%==============================================================================

%==============================================================================
Em se tratando do \textbf{Suggestion 1}, onde é sugerido que as perguntas estejam escritas de maneira que toda a amostra respondente consiga entender e formular uma resposta. 
Já que as perguntas deste questionario se referem a requisitos de software, foi utilizado o modelo de estórias de usuário, em que deixa bem explícito qual o ator, o que se deseja com o determinado requisito e o seu motivo. 
Também foi determinado que as perguntas seriam classificadas como closed questions, que determinam as possíveis respostas do respondente como descrito por \citeonline{forza}. 
Assim, no final de cada página do questionário também continha uma questão open-ended permitindo o respondente dissertar da maneira que bem entender.
%==============================================================================

%==============================================================================
A \textbf{Suggestion 2} se trata da escala utilizada nas perguntas, em um primeiro momento pensou-se em utilizar a escala Likert, mas melhor pensado posteriormente decidiu-se utilizar a escala \ac{MoSCoW}, sendo as possíveis respostas as já presentes no seu próprio nome. 
Ela foi escolhida porque esta mais relacionada a requisitos e serve justamente para a priorização de requisitos de software.
%==============================================================================

%==============================================================================
Em seguida na \textbf{Suggestion 3}, sugere-se que o questionário direcione os participantes para as perguntas que eles possuam mais propriedade para respondê-las, trazendo respostas mais construtivas e relevantes. 
No questionário utilizado, esta divisão esta sendo feita utilizando as perguntas de perfil comentadas na \Cref{survey:profile-questions}, sendo o participante automaticamente direcionado para a seção correspondente com seu perfil.
%==============================================================================

%==============================================================================
Por fim, na \textbf{Suggestion 4} é aconselhado que todas as perguntas que tem um assunto em comum, sejam organizadas próximas umas das outras para facilitar as verificações cruzadas entre as respostas. 
Para implementar esta sugestão, os requisitos estão agrupados por papéis conforme os atores do sistema, sendo eles: 
\begin{inparaenum}[(1)]
    \item Proponente de atividade de extensão;
    \item Instrutor de atividades de extensão;
    \item Coordenador de projetos ou programas de extensão;
    \item Participante de atividades de extensão.
\end{inparaenum}
%==============================================================================
\subsubsection{Feature Suggestions}
%==============================================================================
Para a última página do questionário foi disponibilizado um campo em que os respondentes podem sugerir aos pesquisadores qualquer melhoria, funcionalidade, correção etc. Com estas respostas é possível fazer uma analise qualitativa e conseguir novas ideias para o desenvolvimento e completude da ferramenta final.
%==============================================================================
\subsection{Pilot Questionnaire}\label{sec:survey-pilot}
%==============================================================================
Após ser gerado uma versão estável do questionario, é necessário validá-lo, para isto foi realizado um questionário piloto. 
% \citeonline[p.75]{kasunic2005designing} 
De acordo com \citeonline{kasunic2005designing} a pilot test is a simulation of the real questionnaire carried out with a small number of members from the target audience. 
Para realizá-lo foram escolhidas 7 (sete) pessoas, divididas em: 4 (quatro) alunos de graduação, 2 (dois) professores da \ac{UNIPAMPA} campus Alegrete e 1 (um) \ac{TAE}. 
A escolha dos respondentes do questionario piloto se deu porque ela representa todos os perfis esperados na target sample, e representa a proporção esperada na realização real do questionario.
%==============================================================================

%==============================================================================
Dos escolhidos para a participação do questionário piloto apenas um não conseguiu respondê-lo a tempo, este foi o \ac{TAE}, mas isto no final não foi um problema pois como mencionado na \Cref{sec:survey-sampling}, o questionário está divido em duas partes em que \acp{TAE} e docentes respondem a mesma.
%==============================================================================

%==============================================================================
Com a execução deste piloto foi possível adiquirir diversas sugestões, correções e pontos importantes para a versão final do questionário. 
Um exemplo disso, foi que um participante não se sentiu confortável expondo a sua idade exata, então foi sugerido que esta fosse solicitada utilizando faixas etárias.
%==============================================================================
\subsection{Distribute the Questionnaire}\label{sec:survey-distribute}

O questionário foi distribuído para todas as pessoas que compõem a amostra desta pesquisa. 
Para isto ser executado, primeiro foi realizado uma pesquisa para levantar todos os emails de coordenadores de projetos ou programas de extensão de todos os campus da \ac{UNIPAMPA}, sendo eles os primeiros a responder as respostas do questionario. 
Após 2 (dois) dias, foi enviado emails para todas as secretarias academicas dos campus, solicitando que fosse repassado para todos os seus alunos de todos os cursos. 
Concluindo, ao todo o survey ficou aberto para respostas por 18 (dezoito) dias.

\subsection{Analyze the Results and Write a Report}\label{sec:survey-analyse}

Os resultados quantitativos relacionados a priorização de requisitos, devem ser coletados e organizados em gráficos para melhor entendimento e visualização dos dados. 
Assim será possível ter uma lista ordenada de requisitos que foram considerados mais importantes para os usuários finais.

Em relação as respostas qualitativas, estas serão analisadas caso a caso e se pertinente a sugestão, serão adicionadas dentro do leque final de funcionalidades ou melhorias.

\section{Threats to Validity}\label{sec:survey-threats}

% Decições, problemas, descrever como tentamos contornar a ocorrencia
% Olhar nos trabalhos para exemplos de ameaças
% Piloto, estruturação...

\section{Result Analysis}\label{sec:survey-results}

\subsection{--} % Separação dos participantes, aluno, professor, idade, campus...

\subsection{--} % Participação em algum projeto de extensão

\subsection{--} % Classificação final, graficos       % 27/07
% Extensionly - análise e projeto de software, artefatos da implementação, maior capítulo de todos, modelo de domínio, diagrama de componentes, paradigma de programação, tecnologias, processo da engenharia de software, separação frontend/backend (com mais detalhes técnicos), usar figuras e modelos
%==============================================================================
\chapter{EXTENSIONLY}\label{extensionly}
%==============================================================================

\section{Requirement Engineering}
\subsection{Requirements Elicitation, Modeling and Analysis}
\subsection{User Stories}

\section{Features}
\begin{landscape}
\begin{figure}[!htb]
\caption{Taxonomy of performance testing tools represented by feature model.}
\label{fig:featuremodel}
\begin{adjustbox}{max width=1.5\textwidth}
%% https://tex.stackexchange.com/questions/335708/feature-diagram-in-latex
    \begin{forest}  
    disjunction tree,
    disjuncts from'=1,
    concrete from'=1,
    concrete colour=blue!55!cyan!40,
    abstract colour=blue!85!cyan!15,
    draw colour=darkgray,
    [Performance Testing Process
        [Profiles/Roles, mandatory, l=25mm
            [Performance Engineer, l=20mm
                [Architect, l=20mm]
                [Tester, l=20mm]
                [Analyst, l=20mm]
            ]
        ]
        [Methods, mandatory, l=25mm
            [Scripting, l=20mm]
            [CR, l=20mm]
            [MBT, l=20mm]
        ]
        [Artifacts, mandatory, l=25mm
            [Test Plan, optional, l=20mm]
            [Model, optional, l=20mm]
            [Script, l=20mm]
            [Workload, l=20mm]
            [Scenario, l=20mm]
            [Test Report, l=20mm]
        ]
        [Approaches, mandatory, l=25mm,
            [Load, l=20mm]
            [Stress, l=20mm]
            [Endurance, l=20mm]
            [Spike, l=20mm]
        ]
        [Stages, mandatory, l=71mm
            [Pre-Test, mandatory, l=20mm
                [Planning, optional, l=20mm]
                [Scripting, l=20mm]
                [Design, l=20mm]
                [Configuration, l=20mm]
            ]
            [Test, mandatory, l=20mm
                [Execution, l=20mm]
                [Monitoring, l=20mm]
            ]
            [Post-Test, mandatory, l=20mm
                [Analysis, optional, l=20mm]
                [Reporting, l=20mm]
            ]
        ]
    ]
    \end{forest}
    \end{adjustbox}
    \centering
    \fonte{Author.}
    \end{figure}
\end{landscape}
\subsection{Roles}

\section{Development}
\subsection{Technology Stack}
\subsection{Programming Paradigm}
\subsection{Design Patterns}


\section{Software Architecture}
\subsection{DevOps}
\subsection{Pipeline}

\section{Testing}

\section{Software Artifacts}
\subsection{Domain Model}
\subsection{Component Diagram}
\subsection{Database Schema}
  % 03/08
%==============================================================================
\chapter{PRELIMINARY CONCLUSIONS}\label{conclusions}
%==============================================================================


%------------------------------------------------------------------------------
\section{Dummy}\label{sec:dummy}
%------------------------------------------------------------------------------

% %==============================================================================
% \chapter{Considerações Finais}\label{conclusao}
% %==============================================================================

Em Trabalhos de Conclusão de Curso, use ``\emph{Considerações Finais}'' e não ``\emph{Conclusão}''.

Bom trabalho!
  % 03/08

% ELEMENTOS PÓS-TEXTUAIS
\postextual

% Nome(s) do(s) arquivo(s) .bib (sem a extensão)
% \bibliography{bibliografia}
\printbibliography[title={References}]

\begin{apendicesenv}

% Imprime uma página indicando o início dos apêndices
\partapendices

% Para cada apêndice, um \chapter


\chapter{Survey Questionnaire}\label{appendix:questionnaire}

\includepdf[pages={1-23}]{componentes/5-questionnaire.pdf}

\end{apendicesenv}
 % Apêndices [OPCIONAL]
% \begin{anexosenv}

% Imprime uma página indicando o início dos anexos
% \partanexos

% Para cada anexo, um \chapter

%==============================================================================
% \chapter{Survey Questionnaire}\label{annex:questionnaire}
%==============================================================================




% \end{anexosenv}
 % Anexos [OPCIONAL]
\printindex % Índice Remissivo [OPCIONAL]

\end{document}
